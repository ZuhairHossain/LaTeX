\documentclass[12pt]{article}
\usepackage[english]{babel}
\usepackage[utf8x]{inputenc}
\usepackage{amsmath}
\usepackage{amssymb}
\usepackage{graphicx}
\usepackage{float}
\usepackage[colorinlistoftodos]{todonotes}

\usepackage{fancyhdr}

\pagestyle{fancy}
\fancyhf{}
\fancyhead[LE,RO]{Overleaf}
\fancyhead[RE,LO]{Assignment-01}
\fancyfoot[CE,CO]{\leftmark}
\fancyfoot[LE,RO]{\thepage}

\renewcommand{\headrulewidth}{2pt}
\renewcommand{\footrulewidth}{1pt}



\begin{document}

\begin{titlepage}

\newcommand{\HRule}{\rule{\linewidth}{0.5mm}} % Defines a new command for the horizontal lines, change thickness here

\center % Center everything on the page
 
%----------------------------------------------------------------------------------------
%	HEADING SECTIONS
%----------------------------------------------------------------------------------------

\textsc{\LARGE BRAC University}\\[1.5cm] % Name of your university/college
\includegraphics[scale=.3]{bracu_logo.png}\\[1cm] % Include a department/university logo - this will require the graphicx package
\textsc{\Large Complex Variables \& Laplace Transformations}\\[0.5cm] % Major heading such as course name
\textsc{\large MAT215}\\[0.5cm] % Minor heading such as course title

%----------------------------------------------------------------------------------------
%	TITLE SECTION
%----------------------------------------------------------------------------------------

\HRule \\[0.4cm]
{ \huge \bfseries Assignment-01}\\[0.4cm] % Title of your document
\HRule \\[1.5cm]
 
%----------------------------------------------------------------------------------------
%	AUTHOR SECTION
%----------------------------------------------------------------------------------------

%\begin{minipage}{0.4\textwidth}
%\begin{flushleft} \large
%\emph{Author:}\\
%Syed Zuhair \textsc{Hossain}\\ % Your name
%\end{flushleft}

%\end{minipage}\\[2cm]

% If you don't want a supervisor, uncomment the two lines below and remove the section above
\Large \emph{Author:}\\
Syed Zuhair Hossain\\\textsc{St.ID-19101573\\Section-10 \\ \textbf{SET-A}}\\[2cm] % Your name

%----------------------------------------------------------------------------------------
%	DATE SECTION
%----------------------------------------------------------------------------------------

{\large \today}\\[2cm] % Date, change the \today to a set date if you want to be precise

\vfill % Fill the rest of the page with whitespace

\end{titlepage}
\pagebreak

%%%%%%%%%%%%%%%%%%%%%%%%%%%%%%%%%%%%%%%%%%%%%%%%%%%%%
%%%%%%%%%%%%%%%%%%%%%%%%%%%%%%%%%%%%%%%%%%%%%%%%%%%%%

                %Solution Number 01%
                
%%%%%%%%%%%%%%%%%%%%%%%%%%%%%%%%%%%%%%%%%%%%%%%%%%%%%
%%%%%%%%%%%%%%%%%%%%%%%%%%%%%%%%%%%%%%%%%%%%%%%%%%%%%


\chapter{01}
\textbf{Use the help of the polar representations of complex numbers to express $(1+i)^3$ in the form a + bi, where a and b are real.}

\begin{align*}
    Let,
    z &= 1 + i
    \\
    \therefore r &= \sqrt{1^2 + 1^2}
    \\
    &= \sqrt{2}
    \\
    \theta &= \frac{\pi}{4}
    \\ \\
    \textit{According to De Moivr's theorem,}
    \\
    \therefore z&= r xos\theta + i r sin\theta
    \\
    &= \sqrt{2} cos \frac{\pi}{4} + i \left(\sqrt{2} sin \frac{\pi}{4}\right)
    \\
    \therefore (1+i)^3 &= \left[\sqrt{2} (cos(\frac{\pi}{4}) + i \cdot \sqrt{2} \left(sin \frac{\pi}{4} \right)z \right]^3
    \\
    &= (\sqrt{2})^3 \left(cos\frac{\pi}{4} + i sin \frac{\pi}{4} \right)^3
    \\
    &= 2\sqrt{2} \cdot (e^{i \cdot \frac{\pi}{4}})^3 \hspace{10mm} [\therefore cos\theta + sin\theta = e^{i\theta}]
    \\
    &= 2\sqrt{2} \cdot e^{i \frac{3\pi}{4}}
    \\
    &= 2\sqrt{2} \left(cos \frac{3\pi}{4} + i sin \frac{3 \pi }{4}\right)
    \\
    &= 2\sqrt{2} \left(-\frac{1}{\sqrt{2}} + \frac{1}{\sqrt{2}}\right)
    \\
    &= -\frac{2\sqrt{2}}{\sqrt{2}} + i \cdot \frac{2\sqrt{2}}{\sqrt{2}}
    \\
    &= -2 + 2 i
    \\
    &\hspace{50mm}[Answer]
\end{align*}
\pagebreak

%%%%%%%%%%%%%%%%%%%%%%%%%%%%%%%%%%%%%%%%%%%%%%%%%%%%%
%%%%%%%%%%%%%%%%%%%%%%%%%%%%%%%%%%%%%%%%%%%%%%%%%%%%%

                %Solution Number 02%
                
%%%%%%%%%%%%%%%%%%%%%%%%%%%%%%%%%%%%%%%%%%%%%%%%%%%%%
%%%%%%%%%%%%%%%%%%%%%%%%%%%%%%%%%%%%%%%%%%%%%%%%%%%%%

\chapter{02}
\textbf{Express $e^{2+i \pi 2}$ in the a + bi form.}

\begin{align*}
    Given,
    \\
    e^{2+i\pi 2} &= e^2 \cdot e^{i (\pi 2)}
    \\ \\
    \textit{According to De Moivre's theorem,}
    \\
    z^n &= r^n e^{i n \theta}
    \\
    \textit{By applying this formula here -}
    \\
    z &= e^2 e^{i (2\pi)}
    \\
    \therefore r &= e^2
    \\
    \theta &= 2\pi
    \\
    \textit{we know,}
    \\
    X &= r cos \theta 
    \\
    Y &= r sin \theta
    \\\\
    z &= x + i y
    \\
    &= r cos\theta + i r sin \theta
    \\
    z &= x + iy
    \\
    &= r cos \theta + i r sin \theta
    \\
    &= e^2 cos(2\pi) + i \cdot e^2 \cdot sin(2\pi)
    \\
    &= 7.389 + i \cdot 0
    \\
    &= 7.389
    \\ \\ \\
    &\hspace{50mm}[Answer]
\end{align*}

\newpage
%%%%%%%%%%%%%%%%%%%%%%%%%%%%%%%%%%%%%%%%%%%%%%%%%%%%%
%%%%%%%%%%%%%%%%%%%%%%%%%%%%%%%%%%%%%%%%%%%%%%%%%%%%%

                %Solution Number 03%
                
%%%%%%%%%%%%%%%%%%%%%%%%%%%%%%%%%%%%%%%%%%%%%%%%%%%%%
%%%%%%%%%%%%%%%%%%%%%%%%%%%%%%%%%%%%%%%%%%%%%%%%%%%%%
\chapter{03}
\textbf{Express $\frac{1}{1+i} + \frac{2}{3+2i}$ in the a + bi}

\begin{align*}
    \textit{From the equations we get,}
    \\
    \frac{1}{1+i} &= \frac{1}{1+i} \times \frac{1-i}{1-i}
    \\
    &= \frac{1-i}{1-i^2}
    \\
    &= \frac{1-i}{1-(-1)} \hspace{10mm}[\therefore i^2 = -1]
    \\
    &= \frac{1-i}{2}
    \\
    \\
    \texttt{Again,}
    \\
    \frac{2}{3+2i} &= \frac{2}{3+2i} \times \frac{3-2i}{3-2i}
    \\
    &= \frac{6-4i}{3^2 - (2i)^2}
    \\
    &= \frac{6-4i}{9-(4i)^2}
    \\
    &= \frac{6-4i}{9+4} \hspace{10mm}[\therefore i^2=-1]
    \\
    &= \frac{6-4i}{13}
    \\ \\
    \therefore \frac{1}{1+i} + \frac{2}{3+2i} &= \frac{1-i}{2} + \frac{6-4i}{13}
    \\
    &= \frac{1}{2} - \frac{i}{2} + \frac{6}{13} - \frac{4i}{13}
    \\
    &= \frac{25}{26} - \frac{21i}{26}
    \\ \\
    &\hspace{50mm}[Answer]
\end{align*}
\newpage

%%%%%%%%%%%%%%%%%%%%%%%%%%%%%%%%%%%%%%%%%%%%%%%%%%%%%
%%%%%%%%%%%%%%%%%%%%%%%%%%%%%%%%%%%%%%%%%%%%%%%%%%%%%

                %Solution Number 04%
                
%%%%%%%%%%%%%%%%%%%%%%%%%%%%%%%%%%%%%%%%%%%%%%%%%%%%%
%%%%%%%%%%%%%%%%%%%%%%%%%%%%%%%%%%%%%%%%%%%%%%%%%%%%%

\chapter{04}
\textbf{Find the distinct roots of z for the following equation, z^4 = 3i}

\begin{align*}
    z^4 &= 3i
    \\
    \Rightarrow z^4 &= 0 + 3i
    \\
    \Rightarrow r^4 e^{i4\theta} &= r_0 e^{i \theta_0}
    \\
    \therefore r_0 &= r^4
    \\
    \Rightarrow (r_0)^{\frac{1}{4}} &= r
    \\
    \Rightarrow r &= \left(\sqrt{(3i^2)^2}^{\frac{1}{4}}\right)
    \\
    &= 3^{\frac{1}{4}}
    \\
    \\
    \theta_0 &= 4\theta
    \\
    \Rightarrow 4\theta &= \theta_0 + 2 \pi k
    \\
    &= \frac{\pi}{2\pi k}
    \\
    \therefore \theta &= \frac{\pi}{8} + \frac{\pi k}{2}
    \\
    \texttt{1st root,}
    \\
    z_1 &= (3)^{\frac{1}{4}} \cdot \left[cos(\frac{\pi}{8} + \frac{\pi}{2} \times 0) + isin sin(\frac{\pi}{8} + \frac{\pi}{2} \times 0) \right]
    \\
    &= 1.216 + 0.504 i
    \\
    \\
    \texttt{2nd root,}
    \\
    z_2 &= (3)^{\frac{1}{4}} \left[cos(\frac{\pi}{8} + \frac{\pi}{2} \times (-1)) + i sin(\frac{\pi}{8} + \frac{\pi}{2} \times 1)
    \right]
    \\
    &= -0.504 + 1.216 i
    \\ \\
    \textit{3rd root,}
    \\
    z_3 &= (3)^{\frac{1}{4}} \left[cos(\frac{\pi}{8} + \frac{\pi}{2} \times (-1)) + i sin(\frac{\pi}{8} + \frac{\pi}{2} \times (-1))\right]
    \\
    &= 0.504 - 1.216 i
\end{align*}

\newpage
\begin{align*}
    \textit{4th root,}
    \\
    z_4 &= (3)^{\frac{1}{4}} \left[cos(\frac{\pi}{8} + \frac{\pi}{2} \times 2) + i sin(\frac{\pi}{8} + \frac{\pi}{2} \times 2)
    \right]
    \\
    &= -1.216 - 0.504 i
    \\
    \\
    &\hspace{50mm}[Answer]
\end{align*}
    
\vspace{20mm}
%%%%%%%%%%%%%%%%%%%%%%%%%%%%%%%%%%%%%%%%%%%%%%%%%%%%%
%%%%%%%%%%%%%%%%%%%%%%%%%%%%%%%%%%%%%%%%%%%%%%%%%%%%%

                %Solution Number 05%
                
%%%%%%%%%%%%%%%%%%%%%%%%%%%%%%%%%%%%%%%%%%%%%%%%%%%%%
%%%%%%%%%%%%%%%%%%%%%%%%%%%%%%%%%%%%%%%%%%%%%%%%%%%%%
\chapter{05}
\texttt{Prove by taking z = a + bi, z - \overline{z} = 2iIm z.}

\begin{align*}
    \text{Given that,}
    \\
    z &= a + bi
    \\
    \therefore \overline{z} &= a - b i
    \\ \\
    \therefore L. H. S. &= z - \overline{z}
    \\
    &= a + bi - a + bi
    \\
    &= 2 ib
    \\ \\
    R. H. S. &= 2 i Im (z)
    \\
    &= 2 i Im(a+bi)
    \\
    &= 2 ib
    \\ \\
    \therefore L. H. S. &= R. H. S.
    \\
    \vspace{20mm}
    &\hspace{50mm} [Answer]
\end{align*}

\end{document}