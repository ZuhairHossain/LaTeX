\documentclass[12pt]{article}
\usepackage{amsfonts}
%\usepackage{mathptmx}

\usepackage[utf8]{inputenc}
\usepackage{comment}

%\usepackage{pgfplots}
%\pgfplotsset{width=10cm, compat=1.9}
%\documentclass[border=2mm,tikz]{standalone}
%\usetikzlibrary{datavisualization}

\usepackage[onehalfspacing]{setspace}
\usepackage{fancyhdr}
\usepackage{comment}
\usepackage[a4paper, top=2.5cm, bottom=2.5cm, left=2.5cm, right=2.5cm]%
{geometry}
\usepackage{times}
\usepackage{amsmath}
\usepackage{changepage}
\usepackage{amssymb}
\usepackage{graphicx}

\setcounter{MaxMatrixCols}{30}
\newtheorem{theorem}{Theorem}
\newtheorem{acknowledgement}[theorem]{Acknowledgement}
\newtheorem{algorithm}[theorem]{Algorithm}
\newtheorem{axiom}{Axiom}
\newtheorem{case}[theorem]{Case}
\newtheorem{claim}[theorem]{Claim}
\newtheorem{conclusion}[theorem]{Conclusion}
\newtheorem{condition}[theorem]{Condition}
\newtheorem{conjecture}[theorem]{Conjecture}
\newtheorem{corollary}[theorem]{Corollary}
\newtheorem{criterion}[theorem]{Criterion}
\newtheorem{definition}[theorem]{Definition}
\newtheorem{example}[theorem]{Example}
\newtheorem{exercise}[theorem]{Exercise}
\newtheorem{lemma}[theorem]{Lemma}
\newtheorem{notation}[theorem]{Notation}
\newtheorem{problem}[theorem]{Problem}
\newtheorem{proposition}[theorem]{Proposition}
\newtheorem{remark}[theorem]{Remark}
\newtheorem{solution}[theorem]{Solution}
\newtheorem{summary}[theorem]{Summary}
\newenvironment{proof}[1][Proof]{\textbf{#1.} }{\ \rule{0.5em}{0.5em}}

\newcommand{\Q}{\mathbb{Q}}
\newcommand{\R}{\mathbb{R}}
\newcommand{\C}{\mathbb{C}}
\newcommand{\Z}{\mathbb{Z}}

\begin{document}

\title{MAT120: Integral Calculus and
Differential Equations \\
BRAC University}

\author{Syed Zuhair Hossain \\ St. ID - 19101573 \\ Section - 07 \\ Set-Q \\ Assignment - 04}
\date{\today}
\maketitle

%%%%%%%%%%%Starting Point%%%%%%%%%%%%%%%

%%%%%MATH 01%%%%%%%%%%%%
\section{Evaluate the integral \\[5mm]
        $\int_{1}^{2} \int_{z}^{2} \int_{0}^{\sqrt{3}y} \frac{y}{x^2 + y^2} \ dx \ dy \ dz$
        }

\textbf{Solution \ :}
\begin{align*}
    Let,\\
    &u = \frac{x}{y} & \Rightarrow x^2 = y^2 u^2\\
    &du = \frac{1}{y} dx\\
    \therefore \ &dx = y \ du
\end{align*}

\begin{align*}
    \therefore \ \int \frac{y}{x^2 + y^2} \ dx &= \int \frac{y^2}{y^2 u^2 + y^2 } du \\
    &= \int \frac{y^2}{u^2 + 1} du\\
    &= \int \frac{1}{u^2 +1} du\\
    &= tan^{-1}(u)\\
    &= tan^{-1}\frac{x}{y}
\end{align*}

\begin{align*}
    &\int_{1}^{2} \int_{2}^{3} (\frac{\pi}{3}) dy dx\\[3mm]
    &= \int_{1}^{2} \frac{\pi}{3} \times [y]_{z}^{2} \ dz\\[3mm]
    &= \int_{1}^{2} \frac{\pi}{3} (2-z) dz\\[3mm]
    &= \int_{1}^{2} \frac{2\pi}{3} dz - \int_{1}^{2} \frac{\pi}{3} z dz\\[3mm]
    &= \frac{2\pi}{3} \int_{1}^{2} dz - \frac{\pi}{3} \int_{1}^{2} z dz\\[3mm]
    &= \frac{2\pi}{3} \int_{1}^{2} dz - \left[ \frac{\pi}{3} \times \frac{z^2}{2}\right]_{1}^{2}\\
    &= \frac{2\pi}{3} \int_{1}^{2} dz + (- \frac{4 \pi}{6} + \frac{1 \cdot \pi }{6})\\[3mm]
    &= \frac{2\pi}{3} \int_{1}^{2} dz - \frac{\pi}{2}\\[3mm]
    &= \frac{2 \pi \times 2}{3} - \frac{2\pi \cdot 1}{3} - \frac{\pi}{2}\\[3mm]
    &= \frac{2\pi}{3} - \frac{\pi}{2}\\[3mm]
    &= \frac{4\pi-3\pi}{6}\\[3mm]
    &= \frac{\pi}{6} &[Answer]
\end{align*}
%%%%%%%%%%%%%%%%%%%%%%%
\pagebreak 
\section{Solve}
\centering $(x+1) \frac{dy}{dx} + y = lnx, y(1) = 10$

\renewcommand{\baselinestretch}{2.0}
\begin{align*}
    \textbf{Converting into standard form:}&\\
    & \frac{dy}{dx} + \left(\frac{1}{x+1}\right) y = \frac{ln(x)}{(x+1)} \ \ \ \ \ \ \ \ \ [\textit{dividing by (x+1) in both side}]\\
    \textbf{according to the formula}&\\
    &y{'} + \rho (t) y = g(t)\\
    \therefore \rho(x) &= e^{\int \frac{1}{x+1}dx}\\
    &= e^{ln(x+1)}\\
    &= (x+1)\\\\
    \raggedright\therefore (x+1) y &= \int ln(x)dx\\
    \Rightarrow(x+1)y &= xln(x) - x + c\\
    \therefore y &= \frac{x(ln(x)-x+c)}{x+1}\\
    \textbf{apply the given condition,}\\
    &y(1) = 10\qquad  x=1\\
    &\Rightarrow \frac{1\times ln(1)-1+c}{1+1} = 10\\
    &\Rightarrow \frac{0-1+c}{2} = 0\\
    &c = 21\\
    &\therefore y = \frac{xln(x) - x + 21}{(x+1)}\qquad \textbf{[Answer]}
\end{align*}
\pagebreak

\raggedright\section{Evaluate \qquad $\int_{1}^{4} \int_{0}^{\sqrt{x}}\frac{3}{2} e^{\frac{y}{\sqrt{x}}} dy dx$}

\begin{align*}
    &= \int_{1}^{4} \frac{3}{2} \left[e^{\frac{y}{\sqrt{x}} \times \sqrt{x}}\right]_{0}^{\sqrt{x}} dx\\
    &= \int_{1}^{4} \frac{3}{2} \left[e^1 \cdot \sqrt{x} - e^0 \cdot \sqrt{x}\right] dx\\
    &= \int_{1}^{4} \frac{3}{2} (e\sqrt{x} - \sqrt{x}) dx\\
    &= \int_{1}^{4} \frac{3}{2} \times \sqrt{x} \times (e-1) dx\\
    &= \frac{3}{2} \int_{1}^{4} \sqrt{x} (e-1) dx\\
    &= \frac{3}{2} \times (e-1) \int_{1}^{4}\sqrt{x} dx\\
    &= \frac{3}{2}(e-1) [\frac{2x^\frac{3}{2}}{3}]_{1}^{4}\\
    &= (e-1) [x^{\frac{3}{2}}]_{1}^{4}\\
    &= (e-1) [4^{\frac{3}{2}}-1]\\
    &= 7(e-1)\\
    &\approx 12.0279728 \hspace{100} \textbf{[Answer]}
\end{align*}
\pagebreak

\section{Solve the differential equation using variables separable method : \qquad x^{2} \frac{dy}{dx} =y-xy; y(-1) = -1.}

\begin{align*}
    &\Rightarrow x^2 \frac{dx}{dy} = y(1-x)\\
    &\Rightarrow \frac{1}{y} \frac{dy}{dx} = \frac{1-x}{x*2} = \frac{1}{x^2} - \frac{1}{x}\\
    &\Rightarrow \int y^{-1} dy = \int \frac{(1-x)}{x^2} dx\\
    \textit{Now, integrating both sides,}\\
    &\int \frac{dy}{y} = \int \frac{1}{x^2}dx - \int \frac{1}{x} dx\\
    &\Rightarrow ln(y) = - \frac{1}{x} - ln(x) + c\\\\
    \textbf{Given that,}\\
    &y=-1 \hspace{20} x=-1\\
    &\therefore ln|-1| = \frac{-1}{-1} - ln|-1| + c\\
    &\Rightarrowln(1) = 1- ln(1) + c\\
    &\therefore c=0\\\\
    &\therefore log y = \frac{1}{x} - log(x)-1\\
    &\Rightarrow log(y) + log(x) = - \frac{1}{x} - 1\\
    &\Rightarrow log(yx) = - \frac{1}{x} -1\\
    &\Rightarrow yx = e^{-\frac{1}{x} -1}\\
    &\Rightarrow yx = e^{\frac{-1}{x}} e^{-1}\\
    &\Rightarrow y = \frac{e^{\frac{-1}{x}}}{e^{x}}\hspace{25} \textbf{[Answer]}
\end{align*}

\newpage

\section{Evaluate the integral: \qquad \int_{0}^{\frac{\pi}{4}} \int_{0}^{1} \int_{0}^{x^2} x cosy \ dz \ dx \ dy }

\begin{align*}
    &= \int_{0}^{\frac{\pi}{4}} \int_{0}^{1} [x z cos y]_{0}^{x^2} dx dy\\
    &= \int_{0}^{\frac{\pi}{4}} \int_{0}^{1} (x \cdot x^2 cosy - x \cdot 0 \cdot cosy) dx dy\\
    &= \int_{0}^{\frac{\pi}{4}} \int_{0}^{1} x^3 cosy dx dy\\
    &= \int_{0}^{\frac{\pi}{4}} cosy \int_{0}^{1} x^3 dx dy\\
    &= \int_{0}^{\frac{\pi}{4}} \left[\frac{x^4}{4} cos(y)\right]_{0}^{1}dy\\
    &= \int_{0}^{\frac{\pi}{4}} \left[\frac{x^4}{4} cos(y) - \frac{0}{4} cos(y) \right]dy\\
    &= \int_{0}^{\frac{\pi}{4}} \frac{cosy}{4} dy\\
    &= \frac{1}{4} \int_{0}^{\frac{\pi}{4}} cos \ y \ dy\\
    &= \left[\frac{1}{4} \times sin y\right]_{0}^{\frac{\pi}{4}}\\
    &= \left[\frac{1}{4} \times sin y \right]_{0}^{\frac{\pi}{4}}\\
    &=(\frac{1}{4} sin \frac{\pi}{4} - \frac{1}{4} sin(0))\\
    &= \frac{1}{4} \times \frac{1}{\sqrt{2}} -0\\
    &= \frac{1}{4\sqrt{2}}\\
    \bigbreak
    &\hspace{200} \textbf{[Answer]}
\end{align*}

\pagebreak

\section{Solve the system for x and y in terms of u and v then find the Jacobian $\frac{\partial(x,y)}{\partial(u,v)} .$} 
\centering \bigtitle{\textbf{u = x-y; v = 2x+y}}

\begin{align*}
    Here,\\
    &u = x-y............(1)\\
    &v = 2x+y...........(2)
\end{align*}

\begin{align*}
    \textit{adding equation (1) and (2) we get,}&\\
    & u + v = 3x\\
    & x = \frac{u+v}{3}\\
    & x = \frac{u}{3} + \frac{v}{3}\\
\end{align*}

\begin{align*}
    \textit{Again, extracting equation (2) from (1) we get,}&\\
    & v-u = 2y -x\\
    & 2y = v-u +x\\
    & 2y = v - u + \frac{u}{3} + \frac{v}{3}\\
    & y = \frac{3v-3u+u+v}{3 \times 2}\\
    & y = \frac{4v-2u}{6}\\
    & y = \frac{2v}{3} - \frac{1\cdot u}{3}
\end{align*}
\pagebreak
\begin{align*}
    &\frac{\partial x}{\partial u} =\frac{1}{3} & \frac{\partial y}{\partial u} = -\frac{2}{3}\\
    &\frac{\partial x}{\partial v} =\frac{1}{3} & \frac{\partial y}{\partial u} = \frac{1}{3}\\
\end{align*}

\begin{equation*}
    \textbf{Jacobian,J}
    =\frac{\partial(x,y)}{\partial(u,v)} 
    = \begin{vmatrix}
    \frac{\partial x}{\partial u} & \frac{\partial y}{\partial u}\\ 
    \frac{\partial x}{\partial v} & \frac{\partial y}{\partial u}
    \end{vmatrix}
    = 
    \begin{vmatrix}
    \frac{1}{3} & -\frac{2}{3}\\ 
    \frac{1}{3} & \frac{1}{3}
    \end{vmatrix}
    = \frac{1}{3} \times \frac{1}{3} - \left(-\frac{2}{3} \right) \times \frac{1}{3}
    =\frac{1}{9} + \frac{2}{9} = \frac{1}{3}
\end{equation*}
\bigbreak
\centering \title{\textbf{[Answer]}}
\end{document}
