\documentclass[12pt]{article}
\usepackage{amsfonts}


\usepackage[utf8]{inputenc}
\usepackage{comment}

%\usepackage{pgfplots}
%\pgfplotsset{width=10cm, compat=1.9}
%\documentclass[border=2mm,tikz]{standalone}
%\usetikzlibrary{datavisualization}


\usepackage{fancyhdr}
\usepackage{comment}
\usepackage[a4paper, top=2.2cm, bottom=2.5cm, left=2.2cm, right=2.2cm]%
{geometry}
\usepackage{times}
\usepackage{amsmath}
\usepackage{changepage}
\usepackage{amssymb}
\usepackage{graphicx}%
\setcounter{MaxMatrixCols}{30}
\newtheorem{theorem}{Theorem}
\newtheorem{acknowledgement}[theorem]{Acknowledgement}
\newtheorem{algorithm}[theorem]{Algorithm}
\newtheorem{axiom}{Axiom}
\newtheorem{case}[theorem]{Case}
\newtheorem{claim}[theorem]{Claim}
\newtheorem{conclusion}[theorem]{Conclusion}
\newtheorem{condition}[theorem]{Condition}
\newtheorem{conjecture}[theorem]{Conjecture}
\newtheorem{corollary}[theorem]{Corollary}
\newtheorem{criterion}[theorem]{Criterion}
\newtheorem{definition}[theorem]{Definition}
\newtheorem{example}[theorem]{Example}
\newtheorem{exercise}[theorem]{Exercise}
\newtheorem{lemma}[theorem]{Lemma}
\newtheorem{notation}[theorem]{Notation}
\newtheorem{problem}[theorem]{Problem}
\newtheorem{proposition}[theorem]{Proposition}
\newtheorem{remark}[theorem]{Remark}
\newtheorem{solution}[theorem]{Solution}
\newtheorem{summary}[theorem]{Summary}
\newenvironment{proof}[1][Proof]{\textbf{#1.} }{\ \rule{0.5em}{0.5em}}

\newcommand{\Q}{\mathbb{Q}}
\newcommand{\R}{\mathbb{R}}
\newcommand{\C}{\mathbb{C}}
\newcommand{\Z}{\mathbb{Z}}

\begin{document}

\title{MAT120: Integral Calculus and
Differential Equations \\
BRAC University}

\author{Syed Zuhair Hossain \\ St. ID - 19101573 \\ Section - 07 \\ Set-I}
\date{\today}
\maketitle

%%%%%%%%%%%Starting Point%%%%%%%%%%%%%%%

%%%%%MATH 01%%%%%%%%%%%%
\section{You all have learnt the concept of finding arc lengths of curves that are bounded
over some interval. The formula for finding the aforementioned arc length of a curve, are as follows}


\begin{align*}
    Arc \ Length =  \int_{a}^{b} \sqrt{1+.[f{'}(x)]^2} \ dx
\end{align*}

\begin{paragraph}
where $f{'}(x)$ denotes the first derivative of f(x).\\
Given the function:\\
$$f(x) = 9x^\frac{3}{2}:$$Find out what the arc length is for the function bounded by the interval [0,1] .
\end{paragraph}\\

\textbf{Solution}
\begin{align*}
    Given, f(x) &= 9x^\frac{3}{2}\\
    f{'}(x) &= 27 \frac{\sqrt{x}}{2}\\
            &= \frac{27}{2}\sqrt{x}\\
    \textbf{We know that,} \\
    \texttt{Arc Length = $\int_{a}^{b}  \sqrt{1+.[f{'}(x)]^2} \ dx$}\\
    &= \int_{0}^{1} \sqrt{1+ \left (  \frac{27}{2}\sqrt{x}\right)^2} \ dx\\
    &= \int_{0}^{1} \sqrt{1+ \frac{729}{4}x} \ dx
\end{align*}

\begin{align*}
    let, \\ y&= \frac{729x}{4}+1 \\
    \frac{dy}{dx} &= \frac{729}{4} \Rightarrow dx = \frac{4 dy}{729}\\
    \therefore dy &= \frac{729}{4}dx
    \\ \textit{lower limit,} &\\
    x \rightarrow 0 &; y \rightarrow 1 \\
    \\ \textit{upper limit,} &\\
    y \rightarrow 0 &; y \rightarrow \frac{733}{4}
\end{align*}

\begin{align*}
    \therefore & \int_{0}^{1} \sqrt{1+ \frac{729}{4}x} \ dx\\
    & =  \int_{1}^{\frac{733}{4}} \sqrt{y} \ \frac{4}{729} \ dy \\
    & = \frac{4}{729} \int_{1}^{\frac{733}{4}} \sqrt{y} dy \\
    & = \frac{4}{729} \left [ \frac{2y^\frac{3}{2}}{3} \right ]_{1}^{\frac{733}{4}}\\
    & = \left[ \frac{8y^\frac{3}{2}}{2187}\right]_{1}^{\frac{733}{4}}\\
    & =  \frac{8(\frac{733}{4})^\frac{3}{2}}{2187}- \frac{8(1)^\frac{3}{2}}{2187}\\
    & = \frac{733\sqrt{733}-8}{2187}\\
    & = 9.070517613\\
    & \approx 9.071
\end{align*}

\bigskip
\bigskip
\bigskip

\textbf{So, \ 9.071 is the arc length of $f(x)= 9x^\frac{3}{2}$ for the function bounded by the interval [0,1]}

\pagebreak

%%%%%%%%MATH 02%%%%%%%%%%%%

\section{Find the area of surface that is generated by revolving the portion of the curve $x=2\sqrt{1-y}$ \ , $-1\leq y \leq 0$ about the y-axis.}

\ \ \ \ \ \ \textbf{\large Solution}
\begin{align*}
    Given, \\
    x&=2\sqrt{1-y} \ ,-1\leq y \leq 0; \\
    let,\\
    x&= g(y)\\
    g(y) &= 2 \sqrt{1-y}\\
    g{'}(y) &= 2 \ \ \frac{(-1)}{2\sqrt{1-y}} \ =\  \frac{-1}{\sqrt{1-y}}\\
    g{'}(y)^2 &= \frac{1}{1-y}\\\\
    \sqrt{1+g{'}(y)^2} \ &= \sqrt{1+ \frac{1}{1-y}}\\
                        &= \sqrt{\frac{2-y}{1-y}}
\end{align*}
    
\begin{align*}
\textit{So , here the surface area is}\\
    S &= \int_{-1}^{0} 2\pi 2\sqrt{1-y} \ \frac{\sqrt{2-y}}{\sqrt{1-y}} \ dy \ = \int_{-1}^{0} 4\pi \sqrt{2-y} \ dy\\
    S &= 4 \pi \int_{-1}^{0} \sqrt{2-y} \ dy\\
\end{align*}
\begin{align*}
    \textit{to evaluate this integral, we can make u-substitution,}\\
    Let,&\\
    u&= 2-y\\
    d&u= -dy\\
    \textit{lower limit,} &\\
    y \rightarrow 0 &; u \rightarrow 2 \\
     \textit{upper limit,} &\\
    y \rightarrow -1 &; u \rightarrow 3
\end{align*}

\begin{align*}
    S&= 4 \pi \int_{-1}^{0} \sqrt{2-y} \ dy\\
     &= 4 \pi \int_{2}^{3} u^{-\frac{3}{2}} \ du\\
     &= 4 \pi (\frac{2}{3}) \left [ u^{\frac{3}{2}}\right ]_{2}^{3}\\
     &= \frac{8\pi}{3} \left[(2-y)^{\frac{3}{2}}\right]_{-1}^{0}\\
     &=\frac{8\pi}{3} \left[ (2-(-1))^{\frac{3}{2}} - (2-0)^{\frac{3}{2}} \right]\\
     &= \frac{8\pi}{3} \left(3\sqrt{3} - 2\sqrt{2}\right)\\
     &=19.83580907\\
     &\approx19.84\\\\
     & & \textsc{[Answer]}
\end{align*}

%%%%%%%%MATH 03%%%%%%%%%%%%


\section{Evaluate \ \ $\int_{0}^{\infty} \frac{(x^6-x^3)x^2}{(1+x^3)^5}\ dx$ ; \ $\left[ Use \ \beta (m,n)= \int_{0}^{\infty} \frac{x^{m-1}}{(1+x)^{m+n}} \ dx \right]$}
\textbf{Solution}
\begin{align*}
    let,\\
    x^3 &= u\\
    \Rightarrow 3x^2 &= \frac{du}{dx}\\
    \Rightarrow x^2 dx &= \frac{1}{3} \ du\\
     \textit{lower limit,} &\\
    u\rightarrow 0 &; x \rightarrow 0 \\
     \textit{upper limit,} &\\
    u \rightarrow \infty &; x \rightarrow \infty\\\\\\
    &\therefore \int_{0}^{\infty} \frac{(x^6-x^3)x^2}{(1+x^3)^5} dx\\
    &= \int_{0}^{\infty} \frac{1}{3} \ \times \ \frac{u^2-u}{(1+u)^5} du \\
    &= \frac{1}{3} \left[ \int_{0}^{\infty} \frac{u^{2}}{(1+u)^{5}} du \ - \int_{0}^{\infty} \frac{u}{(1+u)^{5}} du \right]\\
    &= \frac{1}{3} \left[ \int_{0}^{\infty} \frac{u^{3-1}}{(1+u)^{3+2}} du \ - \int_{0}^{\infty} \frac{u^{2-1}}{(1+u)^{3+2}} du \right]\\
    &= \left[\beta(3,2) \ - \ \beta(2,3) \right] & \left[\because \beta(m,n) = \int_{0}^{\infty} \frac{x^{m-1}}{(1+x)^{m+n}}\: dx\right]\\
    &= \left[\frac{\Gamma(3)\Gamma(2)}{\Gamma(3+2)} \ -  \frac{\Gamma(2) \Gamma(3)}{\Gamma (2+3)} \right] & \left[\because \beta(m,n) = \frac{\Gamma(m) \Gamma(n)}{\Gamma (m+n)}  \right]\\
    &= \frac{1}{3} \left[\frac{(3-1)! \ (2-1)!}{(5-1)!} - \frac{(2-1)!\:(3-1)!}{(5-1)!}\right]\\
    &= \frac{1}{3} \left[\frac{2! \ 1!}{4!} \ - \ \frac{1! \ 2! }{4!} \right]\\
    &= \frac{1}{3} \times 0\\
    &= 0 \\
    & &\textsc{[answer]}
\end{align*}

\pagebreak

%%%%%%%%MATH 04%%%%%%%%%%%%


\section{Evaluate $\int_{0}^{\frac{\pi}{2}}\sqrt{tan \theta}$}
\begin{align*}
    let,\\
    I &= \int_{0}^{\frac{\pi}{2}}\sqrt{tan \theta}\\
    I{'}&= \int_{0}^{\frac{\pi}{2}}\sqrt{tan(\frac{\pi}{2}-\theta)} d\theta\\
    &= \int_{0}^{\frac{\pi}{2}}\sqrt{cot\theta} d\theta\\
    2I &= \int_{0}^{\frac{\pi}{2}}(\sqrt{\tan\theta} + \sqrt{\cot\theta})d\theta\ &[I=I{'}]\\
        &= \int_{0}^{\frac{\pi}{2}} \left(\frac{\sqrt{sin\theta}}{\sqrt{cos\theta}} + \frac{\sqrt{cos\theta}}{\sqrt{\sin\theta}}\right)d\theta\\
        &= \int_{0}^{\frac{\pi}{2}} \frac{\sin\theta+\cos\theta}{\sqrt{\sin\theta \cdot \cos\theta}}d\theta\\
        &= \int_{0}^{\frac{\pi}{2}}\frac{\sin\theta+\cos\theta}{\sqrt{\sin2\theta}\times \frac{1}{\sqrt{2}}} d\theta  & \left[\because 2\sin A \cos A= \sin^2A \right]\\
        &= \sqrt{2}\int_{0}^{\frac{\pi}{2}} \frac{\sin\theta + \cos\theta}{\sqrt{1+\sin2\theta-1}}d\theta\\
        &= \sqrt{2}\int_{0}^{\frac{\pi}{2}} \frac{\sin\theta+\cos\theta}{[-(\sin^2\theta+\cos^2\theta)+2\sin\theta\cos\theta + 1]^{\frac{1}{2}}}d\theta\\
        &= \sqrt{2}\int_{0}^{\frac{\pi}{2}} \frac{\sin\theta+\cos\theta}{\sqrt{-(\sin\theta-\cos\theta)^2+1}}d\theta\\
        &= \sqrt{2}\int_{0}^{\frac{\pi}{2}} \frac{\sin\theta+\cos\theta}{\sqrt{1-(\sin\theta-\cos\theta)^2}}d\theta\\
        let, &\\
        &\sin\theta-\cos\theta=t\\
        &(\cos\theta +\sin\theta)d\theta=dt\\
        \textit{lower limit}\\
        &x\rightarrow 0 ;t\rightarrow \sin0-\cos0 =-1\\
        \textit{upper limit}\\
        &x\rightarrow\frac{\pi}{2}; t\rightarrow sin\frac{\pi}{2} - \cos\frac{\pi}{2} = 1\\
        2I&=\sqrt{2} \int_{-1}^{1} \frac{dt}{1-dt^2}\\
          &=\sqrt{2} \left[sin^{-1} t\right]_{-1}^{1}\\ 
          &= \sqrt{2} \left[sin^{-1}(1)- sin^{-1}(-1)\right]\\
          &=\sqrt{2}(\frac{\pi}{2}+\frac{\pi}{2})\\
          &=\sqrt{2} \pi\\
          \therefore I &= \frac{\pi}{\sqrt{2}}\\
          & &\textsc{[answer]}
\end{align*}

%%%%%%%%MATH 05%%%%%%%%%%%%


\section{Evaluate the following indefinite integrals by using appropriate substitutions: $\int \sqrt{4x^2 - 8x + 24} \ dx$}
\textbf{Solution}
\begin{align*}
    Given,&\\
    &\int \sqrt{4x^2 - 8x + 24} \ dx\\
    &= \int \sqrt{4(x^2-2x+6)} \ dx\\
    &= 2\int \sqrt{x^2-2x+6} \ dx\\
    &= 2\int \sqrt{x^2-2x+1+5} \ dx\\
    &= 2\int \sqrt{(x-1)^2+5} \ dx\\
    & &let,\\
    & & u=x-1\\
    & & \frac{du}{dx}=1\\
    & & \therefore du=dx\\
    &= 2\int \sqrt{u^2+5} \ du\\
    &= 2\int \sqrt{u^2+(\sqrt{5})^2} \ du\\
    & &\textit{As we know,}\\
    & &\sqrt{a^2+x^2}=a\tan\theta\\
    & &\textit{So, we can apply trigonometric formula here,}\\
    & &u=\sqrt{tanv}\\
    & &\therefore v=\tan^{-1}\frac{u}{\sqrt{5}}\\
    & & and \ du=\sqrt{5} \sec^2(v) \ dv \\
    &=\int \sqrt{5} sec^2v \times \sqrt{5 \tan^2v+5} \ dv\\
    &=\int \sqrt{5} sec^2v \times \sqrt{5(\tan^2v+1)} \ dv\\
    &=\sqrt{5} \cdot \sqrt{5} \sec^2v \cdot \sec (v) \ dv \\
    &= 5\int \sec^3v \ dv
\end{align*}

\begin{align*}
    &\textit{by applying reduction formula ,}\\
    &=\frac{\sec(v)\tan(v)}{2}+\frac{1}{2} \int sec(v) \ dv\\\\
    &\textit{we know that,}\\
    & \int sec(v) dx = ln(tan(v)+sec(v))+c\\
    &\textit{So, we can write}\\
    &\frac{\ln{\tan(v)+\sec(v)}}{2}+ \frac{5\sec(v)\tan(v)}{2}\\
    & \therefore v=tan^{-1}(\frac{u}{\sqrt{5}})\\
    & \rightarrow tan^{-1}(\frac{u}{\sqrt{5}})=\frac{u}{\sqrt{5}}\\
    &sec^{-1}(\frac{u}{\sqrt{5}})=\sqrt{\frac{u^2}{5}+1}\\
    &= \frac{5ln\left(\sqrt{\frac{u^2}{5}+1}+\frac{u}{\sqrt{5}} \right)}{2} + \frac{\sqrt{5} u \sqrt{\frac{u^2}{5}+1}}{2}\\
    &=\frac{\sqrt{5}\sqrt{\frac{(x-1)^2}{5}+1} \ (x-1)}{2} + \frac{5ln\left( \frac{x-1}{\sqrt{5}} + \sqrt{\frac{(x-1)^2}{5}+1}\right)}{2}\\\\
    &\therefore 2\int \sqrt{x^2-2x+6} \ dx = 2 \left( \frac{\sqrt{5}\sqrt{\frac{(x-1)^2}{5}+1} \ (x-1)}{2} + \frac{5ln\left( \frac{x-1}{\sqrt{5}} + \sqrt{\frac{(x-1)^2}{5}+1}\right)}{2}\right)\\
    &\rightarrow \int \sqrt{4x^2-8x+24} \ dx = \sqrt{5}\sqrt{\frac{(x-1)^2}{5}+1} \ (x-1) + 5ln\left(\sqrt{\frac{x-1}{\sqrt{5}} + 1 }\right) + c\\\\
    & &\textsc{[Answer]}
\end{align*}

\pagebreak

%%%%%%%%%%MATH 06%%%%%%%%%%%%


\section{Evaluate in terms of Gamma function}

\begin{center}
    $\int_{0}^{\infty}x^6 e^{-3x}dx$
\end{center}

\textbf{Solution}

\begin{align*}
    &\int_{0}^{\infty}x^6 e^{-3x}dx\\
    Let,&\\
    u&=3x\\
    \Rightarrow x &=\frac{u}{3}\\
    \frac{d}{dx}(2x)&=\frac{d}{dx}(u)\\
    \Rightarrow 3 &= \frac{du}{dx}\\
    \therefore dx &= \frac{du}{3}
    \\\\
    & \therefore  \int_{0}^{\infty}x^6 e^{-3x}dx\\
    &= \int_{0}^{\infty} (\frac{u}{3})^6 e^{-u} du\\
    &= \int_{0}^{\infty} e^{-u} \ \frac{u^6}{3^6} \ \frac{du}{3}\\
    &= \frac{1}{3^6}\times \frac{1}{3} \int_{0}^{\infty} e^{-u} u^6 du\\
    &= \frac{1}{3^7} \int_{0}^{\infty}e^{-u} u^{7-1} du\\
    &= \frac{1}{3^7}\times \Gamma7 & \left[\therefore \Gamma n= \int_{0}^{\infty} e^{-x} x^{n-1}dx \right]\\
    &=\frac{6!}{3^7}\\
    &=0.329218107\\
    &\approx0.33\\
    & & \textsc{[Answer]}
\end{align*}

\end{document}