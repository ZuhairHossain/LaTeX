\documentclass[12pt]{article}%
\usepackage{amsfonts}


\usepackage[utf8]{inputenc}
\usepackage{comment}

\usepackage{pgfplots}
%\pgfplotsset{width=10cm, compat=1.9}
%\documentclass[border=2mm,tikz]{standalone}
\usetikzlibrary{datavisualization}


\usepackage{fancyhdr}
\usepackage{comment}
\usepackage[a4paper, top=2.2cm, bottom=2.5cm, left=2.2cm, right=2.2cm]%
{geometry}
\usepackage{times}
\usepackage{amsmath}
\usepackage{changepage}
\usepackage{amssymb}
\usepackage{graphicx}%
\setcounter{MaxMatrixCols}{30}
\newtheorem{theorem}{Theorem}
\newtheorem{acknowledgement}[theorem]{Acknowledgement}
\newtheorem{algorithm}[theorem]{Algorithm}
\newtheorem{axiom}{Axiom}
\newtheorem{case}[theorem]{Case}
\newtheorem{claim}[theorem]{Claim}
\newtheorem{conclusion}[theorem]{Conclusion}
\newtheorem{condition}[theorem]{Condition}
\newtheorem{conjecture}[theorem]{Conjecture}
\newtheorem{corollary}[theorem]{Corollary}
\newtheorem{criterion}[theorem]{Criterion}
\newtheorem{definition}[theorem]{Definition}
\newtheorem{example}[theorem]{Example}
\newtheorem{exercise}[theorem]{Exercise}
\newtheorem{lemma}[theorem]{Lemma}
\newtheorem{notation}[theorem]{Notation}
\newtheorem{problem}[theorem]{Problem}
\newtheorem{proposition}[theorem]{Proposition}
\newtheorem{remark}[theorem]{Remark}
\newtheorem{solution}[theorem]{Solution}
\newtheorem{summary}[theorem]{Summary}
\newenvironment{proof}[1][Proof]{\textbf{#1.} }{\ \rule{0.5em}{0.5em}}

\newcommand{\Q}{\mathbb{Q}}
\newcommand{\R}{\mathbb{R}}
\newcommand{\C}{\mathbb{C}}
\newcommand{\Z}{\mathbb{Z}}

\begin{document}

\title{MAT120: Integral Calculus and
Differential Equations \\
BRAC University}

\author{Syed Zuhair Hossain \\ St. ID - 19101573 \\ Section - 07 \\ Set-G}
\date{\today}
\maketitle

\section{Evaluate the following integrals by Integration by Parts}
%Given an experiment and a sample space $\mathcal{S}$, the objective of probability is to assign to each event $\textit{A}$ a number $P(A)$, called the probability of event $\textit{A}$, which will give a precise measure of the chance that $\textit{A}$ will occur. 
%
\subsection{$\int x (tan^{-1}x)dx$}
\textbf{Solution}\\\\
%A computer consulting firm presently has bids out on three projects. Let $A_{i}$= \{awarded \ project \ i\}, for $i=1, 2, 3$, and suppose that $P(A_1)=.22$, $P(A_2)=0.25$, $P(A_3)=.28$, $P(A_1 \cap A_2)= .11$, $P(A_1\cap A_3)=0.05$, $P(A_2\cap A_3)=0.07$, $P(A_1\cap A_2 \cap A_3)=0.01$. Express in words each of the following events, and compute the probability of each event:%

Let, u = x
 \[v = tan^{-1}x\] \\
 du = x dx      \[dv = \frac{1}{x^{2}+1} dx\] \\
$u = \frac{x^{2}}{2}$ \\\\

According to integration by parts formula, we know that, \par \par
%line 1
$ \int u\:v\: dx = u\int vdx - \int \frac{du}{dx}(\int vdx) dx $ \\\\
%line 2
$\Rightarrow \int u\:v\:dx = \frac{x^{2}}{2} \int tan^{-1}x - \int \frac{x^2}{2(x^2+1)}dx $ \\\\
%line 3
$\Rightarrow \int u\:v\:dx = \frac{x^{2}}{2} \int tan^{-1}x - \frac{1}{2}\int (1-\frac{x^2}{2(x^2+1)}) dx $ \\\\
%line 4
$\Rightarrow \int u \:v\:dx = \frac{x^{2}}{2} \int tan^{-1}x - \frac{1}{2} \int 1 \: dx + \frac{1}{2} \int (\frac{1}{x^2+1}) dx$\\\\
%line 5
$\Rightarrow \int u \:v\:dx = \frac{x^{2}}{2} \int tan^{-1}(x) - \frac{x}{2} + \frac{1}{2} \: tan^{-1}\:(x) + c$  \ \ \ \ \ \ \ \ \  $\left [ \because \frac{1}{x^2+1}] = tan^{-1} \:(x) \right ]$\\\\
%line 6
$\therefore \: \: \int u \:v\:dx = \frac{1}{2} (x^2+1)tan^{-1}(x) - x +c \ \ \ \  \left[_{answer}\right]$ \\\\

\pagebreak

\subsection{$\int \sqrt{x} \: ln \: \sqrt{x}\:dx$}

\begin{flushleft}
    $=\int \sqrt{x} \: ln(x^{\frac{1}{2}})\: dx$\\
    $=\frac{1}{2}\int \sqrt{x} \: ln(x)\: dx$\\
\end{flushleft}


Let, \\\\
$u=ln(x)$ \ \ \ \ \ \ \ \ \ \ \ \ \ \ \ \ \ \ $v= \sqrt{x}$ \\\\
$du = \frac{1}{x} \: dx$ \ \ \ \ \ \ \ \ \ \ \ \ \ \ \ \ \ \ $v=\frac{2x^{\frac{3}{2}}}{3} \:dx$

\begin{flushleft}
    According \: to \ integration \ by \ parts \ formula, \ we \ know \ that,\\
    $\int u\:v\: dx &= u\int vdx - \int \frac{du}{dx}(\int vdx) dx$ 
\end{flushleft}


 $\int \sqrt{x} \: ln \: \sqrt{x} \: = ln(x)\: \frac{2}{3}\:(x^{\frac{3}{2}}) \: \frac{1}{3} \: \int \sqrt{x} \: dx $ \\ 
 
$\Rightarrow \int \sqrt{x} \: ln \: \sqrt{x} \: = \frac{1}{3} \: x^{\frac{3}{2}} \: ln(x) - \frac{2}{9} \:x^{\frac{3}{2}} + c$ \\

$\Rightarrow \int \sqrt{x} \: ln \: \sqrt{x} \: = \frac{1}{9} \: x^{\frac{3}{2}} \: (3\:ln(x) - 2) \: + c$ \ \ \ \ \ \ \ \ \ \ \ \ \ \ \ \ [answer]\\


    
\section{Use reduction formula to evaluate: \ \ \ \ \ \
$\int_{0}^{\frac{\pi}{2}} cos^6 (x) \:dx$}

\begin{align*}
    According \: to \ &reduction \ formula \ we \ know \ that, \\
    \int cos^n(x) \: dx &= \frac{1}{n} \: cos^{n-1}(x) sin(x) \: + \frac{n-1}{n} \: \int cos^{n-2}(x) dx \\
    \Rightarrow \int_{0}^{\frac{\pi}{2}} cos^{6} (x)dx &= \frac{1}{6} \int_{0}^{\frac{\pi}{2}} cos ^{6-1}(x)\:sinx \:+  \frac{6-1}{6} \int_{0}^{\frac{\pi}{2}} cos^{6-2}(x)dx \\
    \Rightarrow \int_{0}^{\frac{\pi}{2}} cos^{6} (x) \: dx &= [\frac{1}{6} cos^{5}x\:sinx]_{0}^{\frac{\pi}{2}} \:+\: \frac{5}{6} \int_{0}^{\frac{\pi}{2}} cos^{4}x\:dx \\
    \Rightarrow \int_{0}^{\frac{\pi}{2}} cos^{6}(x) \: dx &= 0 + \frac{5}{6} \int_{0}^{\frac{\pi}{2}} cos^{4}(x) dx \ \ \ \ \ \ [\because \frac{1}{6} cos^{5}(\frac{\pi}{2}) \: sin(\frac{\pi}{2}) \: - \frac{1}{6} cos^{5}(0) sin(0)=0]\\
    \Rightarrow \int_{0}^{\frac{\pi}{2}} cos^{6} (x)\: dx &= \frac{5}{6} (\frac{1}{4} cos^{4-1}x \: sinx +\frac{4-1}{4}\int_{\frac{\pi}{2}}^{0} cos^{4-2}x \:dx)  \ \ \ \ \ \ [\because \ applying \: reduction \: method]\\
    \Rightarrow \int_{0}^{\frac{\pi}{2}} cos^{6}(x) \: dx &= \left [ \frac{5}{24} cos^{3}x\:sinx \right ]_{0}^{\frac{\pi}{2}} \: + \: \frac{5}{8} \int_{0}^{\frac{\pi}{2}} cos^{2}x \: dx\\
    \Rightarrow \int_{0}^{\frac{\pi}{2}} cos^{6} (x)\: dx &= 0 + \frac{5}{8} \int_{0}^{\frac{\pi}{2}}cos^{2}(x)\:dx \ \ \ \ \ \ \left[\because \frac{5}{24} cos^3(\frac{\pi}{2})\:sin(\frac{\pi}{2}) - \frac{5}{24} cos^3(0)sin(0) = 0 \right]\\
    \Rightarrow \int_{0}^{\frac{\pi}{2}} cos^{6} (x)\: dx &= \frac{5}{8} \int_{0}^{\frac{\pi}{2}}cos(2x) + \frac{1}{2} \int_{0}^{\frac{\pi}{2}}1\:dx\\
\end{align*}

\pagebreak

\begin{align*}
    let,\\
    u &= 2x \\
    \because du &= 2 \: dx \\
    So,\: we \: get \: &a \: new \: lower \: bound \: from \: here, \\
    u &=2 \times 0 \\
      &= 0\\
      and \: in \: &upper \: bound \\
      u &= 2 \times \frac{\pi}{2} \\
        &= \pi \\
        So,\\
        &\frac{5}{32} \int_{0}^{\pi} cos(u) du+ \frac{5}{16} \int_{0}^{\frac{\pi}{2}}1dx\\
\end{align*}

\begin{align*}
    Applying \ fundamental \ &theorem \ of \ calculus\\
    \left[\frac{5sin(u)}{32} \right]_{0}^{\pi} &= \frac{5sin(\pi)}{32} \: - \frac{5sin(0)}{32}\ \ \ \ \  \ \left[\therefore \int cos(u)\: = \: sin(u)\right]\\
    \left[\frac{5sin(u)}{32} \right]_{0}^{\pi} &=0\\
\end{align*}

\begin{align*}
    So, \ we \ get \ \ \ \ \ \ &\\
    \Rightarrow \int_{0}^{\frac{\pi}{2}} cos^{6}(x) \: dx &= \frac{5}{16} \int_{0}^{\frac{\pi}{2}} \: 1 \: dx \\
    \Rightarrow \int_{0}^{\frac{\pi}{2}} cos^{6}(x) \: dx &= \left[ \frac{5x}{16}\right]_{0}^{\frac{\pi}{2}} \\
    \Rightarrow \int_{0}^{\frac{\pi}{2}} cos^{6}(x) \: dx &= \frac{5 \pi}{32} \: - \: \frac{5\times0}{32}\\
    \Rightarrow \int_{0}^{\frac{\pi}{2}} cos^{6}(x) \: dx &= \frac{5\pi}{32}  &  [answer]\\
\end{align*}

\pagebreak

\section{Evaluate the integral using appropriate substitution: \\ \ \ \ \ \ \ \ \ \ \ \ \ \ \ \ $\int \frac{cos4\theta}{1+2sin\:4\theta} d\theta$}

\begin{align*}
    let, \ \ \ \ \ &\\
    u &= 4 \theta \\
    du &= 4\: d\theta\\
    So, \ we \ get \ \ \ \ &\\
    \frac{1}{4} &\int \frac{cos(u)}{1+2sin(u)}du\\
    y &= 2sin(u)+1\\
    dy &= 2cos(u)du\\
    So, we \ get\\
    \frac{1}{8} &\int \frac{1}{y} \ dy \\
    &= \frac{1}{8} \: \times \ ln(y) +c \\
    &= \frac{1}{8} \: ln(2sin(u)+1)+c \\
    &= \frac{1}{8} \: ln (2sin(4\theta)+1)+c & [answer]\\
\end{align*}


\section{Use Gamma Function to evaluate \ $\int_{0}^{\infty} x^{5} \ e ^{-x^2} \ dx$}

\begin{align*}
    let, \ \ \ \ \ \ \ &\\
    x^2 &= u \\
    \Rightarrow \frac{d}{dx} (x^2) &= \frac{d}{dx}(u)\\
    \Rightarrow 2x &= \frac{du}{dx} \\
    \Rightarrow x\:dx &= \frac{du}{2}\\
    \therefore dx &= \frac{du}{2x} = \frac{du}{2\sqrt{u}}\\\\
    So,  \ \ \ \ \ \ \ \ \ \ \ \ \ \ \ \ \ \ \ \ \   &\\
    \int_{0}^{\infty} x^{5} \ e ^{-x^2} \ dx &=  \int_{0}^{\infty} (\sqrt{u} \: )^5 \:\cdot e^{-u} \cdot \frac{du}{2\sqrt{u}}\\
    \int_{0}^{\infty} x^{5} \ e ^{-x^2} \ dx &= \int_{0}^{\infty} u^{\frac{5}{2}} \cdot e^{-u} \cdot \frac{du}{2\:u^{\frac{1}{2}}} \\
    \int_{0}^{\infty} x^{5} \ e ^{-x^2} \ dx &=  \frac{1}{2} \int_{0}^{\infty} e^{-u} \cdot u^{\frac{5-1}{2}}du\\
    \int_{0}^{\infty} x^{5} \ e ^{-x^2} \ dx &=  \frac{1}{2} \int_{0}^{\infty} e^{-u} \cdot u^{2}\:du\\
    \int_{0}^{\infty} x^{5} \ e ^{-x^2} \ dx &=  \frac{1}{2} \int_{0}^{\infty} e^{-u} \cdot u^{3-1}du\\
    \int_{0}^{\infty} x^{5} \ e ^{-x^2} \ dx &= \frac{1}{2} \times \Gamma{3}\ \ \ \ \ \ \ \ \ \ \left[ \therefore \int_{0}^{\infty} \: e^{-u} \: u^{n-1} \: du \: = \Gamma{n}\right] \\
    \int_{0}^{\infty} x^{5} \ e ^{-x^2} \ dx &= \frac{1}{2} \times 2! \\
    \int_{0}^{\infty} x^{5} \ e ^{-x^2} \ dx &= 1 \ \ \ \ \ \ \ \ \ \ \ \ \ \ \ \ \ \ \ \ \ \ \left[answer \right]\\
\end{align*}

\pagebreak

\section{Prove the Fundamental theorem of Calculus (It cannot be an exact copy from
any source. You can use references but it should be properly cited. Also, you
cannot copy paste directly. In that case you won’t receive any marks).}

\begin{parag}
    According to Fundamental Theorem of Calculus we know that, if there is a graph which occupies a place say, A under the graph of f which is continuous  [a,b] and F is an anti derivative of f, then\\
    \centering $\int _{a}^{b} f(x) dx = F(b) \: - \: F(a)$\\
\end{parag}

\begin{align*}
    Let, \ we \ are \ &given \ an \ \ function \ f(t) \\
    define \ the \ &function \ F(x) \ - \\
    F(x) &= \int_{a}^{x} f(t) dt\\
    \\
    Say, \ x_1 \ and \ \Delta x \ &are \ two \ values\ between \ [a+b] \\
    So, \ we \ get, \\
    F(x_1) &= \int_{a}^{x_1} \ f(t) \ dt;\\
    and\\
    F(x_1 + \Delta x) &= \int_{a}^{x_1 + \Delta x} f(t) dt\\
    If \ we \ substract, \ then \ we \ get, &\\
    \int_{a}^{x_1+ \Delta x} f(t)dt - \int_{a}^{x_1} f(t) dt &=  \int_{x_1}^{x_1+ \Delta x} f(t)dt\\
    F(x_1 + \Delta x) - F(x_1) &= \int_{x_1}^{x_1+ \Delta x} f(t)dt\\\\
    According \ to \ mean \ & value \ theorem,\\
    A \ real \ number \ exists \ &between \ r \in [x_1, x_1+\Delta] \\ 
    \int_{x_1}^{x_1+ \Delta x} f(t)dt &= f(r) \cdot \Delta x\\
    F(x_1 + \Delta x)- F(x_1) &= f(r) \cdot \Delta x\\
    \frac{F(x_1 + \Delta x)- F(x_1)}{\Delta x} &= f(r) \cdot \Delta x\\
    \lim_{\Delta x \to 0} \frac{F(x_1 + \Delta x)- F(x_1)}{\Delta x} &= \lim_{\Delta x \to 0} f(r) \\
    {F}'(x_1) &= \lim_{\Delta x \to 0} f(r) \ \ \ \ \ \ \ \left[\therefore \lim_{\Delta x \to 0} \frac{F(x_1 + \Delta x)- F(x_1)}{\Delta x}={F}'(x_1) \right]\\
\end{align*}

\newpage

\begin{align*}
    So, \lim_{\Delta x \to 0}x_1 \ &= x_1 \\
    and \lim_{\Delta x \to 0}x_1 + \Delta x \ &= x_1\\
   \therefore \lim_{\Delta x \to 0}r \ &= x_1\\
   We \ can \ now \ say \ that, \ & \ f \ is \ continuous \ at \ r,\\
   \therefore {F}'(x_1) = f(x_1)\\\\
   \textbf{[proved]}\\
\end{align*}

\begin{thebibliography}{9}
\bibitem{latexcompanion} 
Howard A.,Iril B., Stephen D., 
\textit{Calculus}. 
10th ed.,John Willey and Sons, United States of America, 2012.

\bibitem{Spivak} 
Spivak, Michael (1980), Calculus (2nd ed.), Houston, Texas: Publish or Perish Inc.

\bibitem{Wikipedia} 
Wikipedia : Fundamental Theorem of Calculus.
\\\texttt{https://en.wikipedia.org/wiki/Fundamental\_theorem\_of\_calculus}
\end{thebibliography}

\bigskip 

\section{You are asked to integrate the following definite integrals,\\ $\int_{0}^{2\pi} sinx\:dx$}

\begin{align*}
    (a)\\
    \int_{0}^{2\pi} sinx\:dx &= \left[-cos\:x \right]_{0}^{2\pi}\\
                             &= -\:\left[cos\:x \right]_{0}^{2\pi}\\
                             &= -[cos2\pi \: - \: cos(0)]\\
                             &= -(1-1)\\
                             &= 0\\
\end{align*}


\begin{parag}
    According to Fundamental Theorem of Calculus  we know that, we will get an area after integration and area has a space which is obviously greater than zero.\\
    But, here we are getting zero, because the value of $cos2\pi$  is  positive and the value of cos(0) is negative. So,they are neglecting each other and we are getting a linear value which is zero.
\end{parag}

\centering
\pgfkeys{/pgfplots/Axis Style/.style={
    width=13.5cm, height=5cm,
    axis x line=center, 
    axis y line=middle, 
    samples=100,
    ymin=-1.5, ymax=1.5,
    xmin=-7.0, xmax=7.0,
    domain=-2*pi:2*pi
}}

%\begin{document}
\begin{tikzpicture}
\begin{axis}[
    Axis Style,
    xtick={
        -6.28318, -4.7123889, -3.14159, -1.5708,
        1.5708, 3.14159, 4.7123889, 6.28318
    },
    xticklabels={
        $-2\pi$, $-\frac{3\pi}{2}$, $-\pi$, $-\frac{\pi}{2}$,
        $\frac{\pi}{2}$, $\pi$, $\frac{3\pi}{2}$, $2\pi$
    }
]
\addplot [mark=none, ultra thick, blue] {sin(deg(x))};
\addlegendentry{$sin(x)$}
\end{axis}
\end{tikzpicture}

\centering
\begin{align*}
    &\ \ \  \ \ \ (b)\\
    \int_{0}^{2\pi} \left | sinx \right |\:dx &= \int_{0}^{\pi} +\:sinx\:dx \: + \: \int_{\pi}^{2\pi} -\: sinx\:dx\ \ \ \ \ \ \ \ \ \ \ \ \left[ plotting \ the \ value \ in \ graph\right]\\
    &= \int_{0}^{\pi} \:sinx\:dx \: - \int_{\pi}^{2\pi} +\:sinx\:dx \\
    &= \left[-cosx\right]_{0}^{\pi} - \left[-cosx \right]_{\pi}^{2\pi}\\
    &= -\left[cosx\right]_{0}^{\pi} - \left[-cosx \right]_{\pi}^{2\pi}\\
    &= -\left[cos\pi \: - \: cos0\right] - \left[-cos2\pi \: - cos\pi \right]\\
    &= -\left[-1-1\right] + \left[1-(-1)\right]\\
    &= 2 \: + 2\\
    &= 4\\
    & \ \ \ \ \ \ \ \ \ \ \ \ \ \ \ \ \ \ \ \ \ \ \ \ \ \ \ \ \ \ \ \ [answer]\\
\end{align*}

\end{document}