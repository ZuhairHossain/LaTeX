\documentclass{article}

\usepackage{fancyhdr}
\usepackage{extramarks}
\usepackage{amsmath}
\usepackage{amssymb}
\usepackage{amsthm}
\usepackage{amsfonts}
\usepackage{tikz}
\usepackage[plain]{algorithm}
\usepackage{algpseudocode}

\usetikzlibrary{automata,positioning}

%
% Basic Document Settings
%

\topmargin=-0.45in
\evensidemargin=0in
\oddsidemargin=0in
\textwidth=6.5in
\textheight=9.0in
\headsep=0.25in

\linespread{1.1}

\pagestyle{fancy}

\renewcommand\headrulewidth{0.4pt}
\renewcommand\footrulewidth{0.4pt}

\setlength\parindent{0pt}

%
% Create Problem Sections
%


\setcounter{secnumdepth}{0}
\newcounter{partCounter}
\newcounter{homeworkProblemCounter}
\setcounter{homeworkProblemCounter}{1}
\nobreak\extramarks{Problem \arabic{homeworkProblemCounter}}{}\nobreak{}



\newcommand{\hmwkTitle}{Monthly Assignment\ \#Set-D}
\newcommand{\hmwkDueDate}{\textbf{Multiple Author}}
\newcommand{\hmwkClass}{MAT120}
\newcommand{\hmwkClassTime}{}
\newcommand{\hmwkClassInstructor}{\item Tarannaum Jahan Sultan  [ Sec-05, ID-19101348 ]\\[5mm]
Sanzida Akter  [ Sec-06, ID-19101584 ]\\[5mm]
Kaosar Ahmed  [ Sec-18, ID-19101328 ]\\[5mm]
Md Aminul Haque  [ Sec-08,  ID-19101580 ]\\[5mm]
Syed Zuhair Hossain  [ Sec-07, ID-19101573 ]
}
\newcommand{\hmwkAuthorName}{\textbf{BRAC University}}

%
% Title Page
%

\title{
    \vspace{2in}
    \textmd{\textbf{\hmwkClass:\ \hmwkTitle}}\\
    \normalsize\vspace{0.1in}\small{ \hmwkDueDate}\\
    \vspace{0.1in}\large{\textit{\hmwkClassInstructor\ \hmwkClassTime}}
    \vspace{3in}
}

\author{\hmwkAuthorName}
\date{}

\renewcommand{\part}[1]{\textbf{\large Part \Alph{partCounter}}\stepcounter{partCounter}\\}

%
% Various Helper Commands
%

% Useful for algorithms
\newcommand{\alg}[1]{\textsc{\bfseries \footnotesize #1}}

% For derivatives
\newcommand{\deriv}[1]{\frac{\mathrm{d}}{\mathrm{d}x} (#1)}

% For partial derivatives
\newcommand{\pderiv}[2]{\frac{\partial}{\partial #1} (#2)}

% Integral dx
\newcommand{\dx}{\mathrm{d}x}

% Alias for the Solution section header
\newcommand{\solution}{\textbf{\large Solution}}

% Probability commands: Expectation, Variance, Covariance, Bias
\newcommand{\E}{\mathrm{E}}
\newcommand{\Var}{\mathrm{Var}}
\newcommand{\Cov}{\mathrm{Cov}}
\newcommand{\Bias}{\mathrm{Bias}}

\begin{document}

\maketitle

%%%%%%%%%%%%%%%%%%%%%%%%%%%%%%%%%%%%%%%%%%%%%%%%%%%%%
%%%%%%%%%%%%%%%%%%%%%%%%%%%%%%%%%%%%%%%%%%%%%%%%%%%%%

                %Solution Number 01%
                
%%%%%%%%%%%%%%%%%%%%%%%%%%%%%%%%%%%%%%%%%%%%%%%%%%%%%
%%%%%%%%%%%%%%%%%%%%%%%%%%%%%%%%%%%%%%%%%%%%%%%%%%%%%
\begin{homeworkProblem}
    \[      \textbf{\underline{Answer to the Question Number One}}
    \]
    \textbf{Part (a)}
    \\
    \[
    \begin{split}
        Given,&
        \\
        P(a \leq x \leq b) &= \int_{a}^{b} f(x) dx 
        \\
        f(x) &=
        \begin{cases}
                0 , \ \ \ \ \ \ \ \ \ \  x \leq 0 
                \\
                kc^{-kx} , \ \ \ \ x \geq 0 
        \end{cases}
        \\ \\
        \textit{Since accidents are occurring at a rate one of every 3 months,}
        \\
         k &= 3 
        \\
        \therefore f(x) &= k e^{-kx}
        \\
        &= 3 e^{-3x}
    \end{split}
    \]
\end{homeworkProblem}
\\ \\ \\ \\ \\ 

\begin{homeworkProblem}
    \textbf{Probability of no accident during 8 month interval is,} \\
    \[
    \begin{split}
        P(x \geq 8) &= 1 - (8 \geq x \geq 0) 
        \\
        &= 1 - \int_{0}^{8} 3 e^{-3x} dx
        \\
        &= 1 - 3 \left[\frac{e^{-3x}}{-3} \right]_{0}^{8}
        \\
        &= 1 + \left[e^{-3x} \right]_{0}^{8}
        \\
        &= 1 + \left[e^{24} - e^0 \right]
        \\
        &= 1 + e^{24} - 1
        \\
        &= 3.8 \times 10^{-11}
    \end{split}
    \\
    \\
    \\ \\ \\ \\
    \centering\textbf{The Probability or chance of no accident in 8 month is 3.8 $\times 10^{-11}$.}
    \\
    \centering\textbf{Since, the value is very small, it can be said that the changes were effective.
    }
    \]
\end{homeworkProblem}
\pagebreak

%%%%%%%%%%%%%%%%%%%%%%%%%%%%%%%%%%%%%%%%%%%%%%%%%%%%%
%%%%%%%%%%%%%%%%%%%%%%%%%%%%%%%%%%%%%%%%%%%%%%%%%%%%%

                %Solution Number 02%
                
%%%%%%%%%%%%%%%%%%%%%%%%%%%%%%%%%%%%%%%%%%%%%%%%%%%%%
%%%%%%%%%%%%%%%%%%%%%%%%%%%%%%%%%%%%%%%%%%%%%%%%%%%%%
\begin{homeworkProblem}
    \[      \textbf{\underline{Answer to the Question Number Two}}
    \]
    \textbf{Part (a)}
    \\
    \[
    \begin{split}
    &d[HBr] = K [H_2][Br_2]^{\frac{1}{2}}
    \\
    \textit{for this reaction, Differential Equation comes,}
    \\
    \frac{dx}{dt} = k(a-x)(b-x)^{\frac{1}{2}}
    \\
    where,
    \\
    x = [HBr] \ \& \ a,b \ are \ the \ initial \ concentrations \ of \ hydrogen \ and \ bromine
    \end{split}
    \]
    \\
    \[
    \begin{split}
        \textbf{(a)}
        \\
        \frac{dx}{dt} &= k (a-x)(b-x)^{\frac{1}{2}}
        \\
        &= k(a-x)^{\frac{3}{2}}
        \\
        \Rightarrow \frac{dx}{(a-x)^{\frac{3}{2}}} dx &= k dt
        \\
    \end{split}
    \]
    \\
    \[
    \begin{split}
        Let,\\
        u = a-x\\
        \Rightarrow du = -dx\\
        \Rightarrow -dx = du\\
        \therefore dx = -du\\\\
      \textbf{Integrating both sides and plugging the value}
      \\\\
      &\Rightarrow - \int \frac{1}{u^{\frac{3}{2}}} du = \int k dt
      \\
      &\Rightarrow \frac{-u^{-\frac{1}{2}}}{-\frac{1}{2}} = \int k dt
      \\
      &\Rightarrow 2u^{-\frac{1}{2}} = kt + c
      \\
      &\Rightarrow 2 (a-x)^{-\frac{1}{2}} = kt + c
      \\
      &\Rightarrow (a-x)^{\frac{1}{2}} = \frac{kt + c}{2}
      \\
      &\Rightarrow \sqrt{a-x} = \frac{2}{kt + c}
      \\
      &\Rightarrow x = a-\frac{4}{(kt+c)^2}
    \end{split}
    \]
    \pagebreak
    \[
    \begin{split}
        &Now,
        \\
        &x(0) = 0
        \\
        So,
        \\
        x=0, \ \ \ \ \ \ & \ \ \ \ \ t = 0
        \\
        x &= a-\frac{4}{(kt+c)^2}
        \\
        \Rightarrow 0 &= a - \frac{4}{c^2}
        \\
        \Rightarrow a &= \frac{4}{c^2}
        \\
        \Rightarrow c &= \sqrt{\frac{4}{a}}
        \\
        c &= \frac{2}{\sqrt{a}}
        \\\\
        \therefore x &= a - \frac{4}{\left(kt + \frac{2}{\sqrt{a}} \right)^2}
        \\ \\ \\ \\ \\ \\ \textbf{[Answer]}
    \end{split}
    \]
\end{homeworkProblem}
\\\\
\begin{homeworkProblem}
    \textbf{Part=b}
    \\
    \[
    \begin{split}
        \textbf{The differential equation is}&
        \\
        &\frac{dx}{dt} = k (a-x)(b-x)^{\frac{1}{2}}
        \\
        &\Rightarrow \frac{dx}{(a-x)(b-x)^{\frac{1}{2}}} = kdt
        \\
        &\Rightarrow \int \frac{dx}{(a-x) \sqrt{b-x}} = \int k dt\ \ \ \ \ \ \ \ \  \ \ \textit{[Integrating both sides]}
        \\\\\\
        Let,
        \\
        &u = \sqrt{b-x}
        \\
        &\Rightarrow u^2 = b-x
        \\
        &\Rightarrow x = b- u^2
        \\
        &\Rightarrow dx = - 2 u du
        \\
        &\Rightarrow - dx = 2 u du
    \end{split}
    \]
    \pagebreak
    \[
    \begin{split}
        & \Rightarrow - 2 \int \frac{1}{(a-b+u^2)} du = kt + c
        \\
        & \Rightarrow \frac{1}{(\sqrt{a-b})^2 + u^2} du = kt + c
        \\
        & \Rightarrow - \frac{2}{\sqrt{a-b}} tan^{-1} \frac{u}{\sqrt{a-b}} = kt + c  \ \ \ \ \ \ \ \ \ \ \ 
        \left[\because \int \frac{1}{a^2 - u^2}du = \frac{1}{a}tan^{-1} (\frac{u}{a})+c \right]
        \\
        &\Rightarrow - \frac{2}{\sqrt{a-b}} tan^{-1} \sqrt{\frac{b-x}{a-b}} = kt + c
        \\\\
        &x(0) = 0
        \\
        So,
        \\
        &x=0 \ \ \ \ \ \ \ \ \ \ t=0
        \\
        &\Rightarrow \frac{-2}{\sqrt{a-b}} tan^{-1} \sqrt{\frac{b-0}{a-b}} = k \cdot (0) + c 
        \\
        &\Rightarrow \frac{-2}{\sqrt{a-b}} tan^{-1} \sqrt{\frac{b}{a-b}} = c
        \\\\\\
        & \textbf{Therefore,}
        \\
        &\frac{-2}{\sqrt{a-b}} tan^{-1} \sqrt{\frac{b-x}{a-b}} = kt - \frac{2}{\sqrt{a-b}} tan^{-1} \sqrt{\frac{b}{a-b}}
        \\
        &\therefore t = \frac{2}{k\sqrt{a-b}}
        \\ \\ \\ \\
        &\ \ \ \ \ \ \ \ \ \ \ \ \ \ \ \ \ \ \ \  \ \ \ \ \ \ \ \ \ \ \ \ \ \ \ \textbf{[Answer]}
    \end{split}
    \]
\end{homeworkProblem}
\pagebreak

%%%%%%%%%%%%%%%%%%%%%%%%%%%%%%%%%%%%%%%%%%%%%%%%%%%%%
%%%%%%%%%%%%%%%%%%%%%%%%%%%%%%%%%%%%%%%%%%%%%%%%%%%%%

                %Solution Number 03%
                
%%%%%%%%%%%%%%%%%%%%%%%%%%%%%%%%%%%%%%%%%%%%%%%%%%%%%
%%%%%%%%%%%%%%%%%%%%%%%%%%%%%%%%%%%%%%%%%%%%%%%%%%%%%
\begin{homeworkProblem}
    \[      \textbf{\underline{Answer to the Question Number Three}}
    \]
    (a)

\begin{align*}
    Given\:\:that,\\
    & m \frac{dv}{dt}=mg-cv\\
    &\Rightarrow m \frac{dv}{mg-cv}=dt\\
    &\Rightarrow m \int \frac{dv}{mg-cv}=\int dt\\
    &\Rightarrow -\frac{m}{c}ln|mg-cv|=t+C \\
    &\Rightarrow -\frac{m}{c}ln(mg-cv)=t+C\:\:.....(1)\\
\end{align*}
As the object dropped from rest, here-
\par
\vspace{5mm}
\hspace{60mm}
v=0 and t=0
\begin{align*}
    So,\\
    &\Rightarrow -\frac{m}{c}\:ln\:mg=C \\
    \vspace{5mm}
    &Now,\:\:putting\:\:value\:\:of\:\:C\:\:in\:\:Eq.(1)\\
    \vspace{5mm}
    &\Rightarrow -\frac{m}{c}\:ln\:(mg-cv)=t-\frac{m}{c}ln\:mg\\
    &\Rightarrow \frac{m}{c}\:ln\:\frac{mg}{(mg-cv)}=t\\
    &\Rightarrow ln\:\frac{mg}{(mg-cv)}=\frac{c}{m}t\\
    &\Rightarrow \frac{mg}{(mg-cv)}=e^{ct/m}\\
    &\Rightarrow mg-cv=\frac{mg}{e^{ct/m}}\\
    &\Rightarrow mg-cv=mge^{-ct/m}\\
    &\Rightarrow cv=mg-mge^{-ct/m}\\
    &\Rightarrow v=\frac{mg}{c}(1-e^{-ct/m})\\
    So,\\
    &v=\frac{mg}{c}(1-e^{-\frac{c}{m}t})\:\:\:\:\:\:[Solved]\\
\end{align*}

\newpage
\vspace{5mm}
\hspace{20mm}
\large
(b)
\par
\vspace{5mm}
\hspace{40mm}
Given that,
\par
\hspace{70mm}
v(t)=s'(t)
\begin{align*}
    So,\\
    &s(t)=\int_{0}^{t}\:v(\tau)d\tau\\
    We\:\:have\:\:found\:\:from \:(a),\\
    &v=\frac{mg}{c}(1-e^{-\frac{c}{m}t})\\
    putting\:\:the\:\:value\:\:of\:\:v,\\
    &s(t)=\int_{0}^{t}\:\frac{mg}{c}(1-e^{-\frac{c}{m}\tau})\:dt\\
    &= \frac{mg}{c}\:{[\tau+\frac{m}{c}e^{-\frac{c}{m}\tau}]}_{0}^{t}\\
    &=\frac{mg}{c}t+\frac{m^2 g}{c^2}e^{-\frac{c}{m}t}-\frac{m^2 g}{c^2}\\
\end{align*}
\par
\hspace{110mm}
\Large Ans.

\end{homeworkProblem}
\pagebreak


%%%%%%%%%%%%%%%%%%%%%%%%%%%%%%%%%%%%%%%%%%%%%%%%%%%%%
%%%%%%%%%%%%%%%%%%%%%%%%%%%%%%%%%%%%%%%%%%%%%%%%%%%%%

                %Solution Number 04%
                
%%%%%%%%%%%%%%%%%%%%%%%%%%%%%%%%%%%%%%%%%%%%%%%%%%%%%
%%%%%%%%%%%%%%%%%%%%%%%%%%%%%%%%%%%%%%%%%%%%%%%%%%%%%

\begin{homeworkProblem}
    \[      \textbf{\underline{Answer to the Question Number Four}}
    \]
    
    \textbf{Part (a)}
    \[
    \begin{split}
        Given,
        \\
        &\int_{-\infty}^{+\infty} \rho (x) = 1
        \\
        & \therefore \rho (x) = A e^{-\lambda (x-a)^2}
        \\ \\
        \textbf{As, A, a and $\lambda$ are constants}
        \\
        let,
        \\
        &u = \lambda (x-a)
        \\
        &u^2 = (\sqrt{\lambda})^2(x-a)^2
        \\
        \therefore \frac{du}{dx} = \sqrt{\lambda}
        \\
        \Rightarrow dx = \frac{du}{\sqrt{\lambda}}
    \end{split}
     \]
     \biggap
     \[ 
     \begin{split}
         & \int_{-\infty}^{+\infty} \rho (x) = 1
         \\
         & \Rightarrow \int_{-\infty}^{+\infty} A e^{-\lambda (x-a)^2} =1
         \\
         & \Rightarrow A \int_{-\infty}^{+\infty} e^{-u^2} \frac{du}{\sqrt{\lambda}} =1 \ \ \ \ \ [Plugging \ the \ value]
         \\
         & \Rightarrow \frac{A}{\sqrt{\lambda}} \int_{-\infty}^{+\infty} e^{-u^2} du = 1
         \\
         &\Rightarrow \frac{A}{\sqrt{\lambda}} \times \sqrt{\pi} = 1
         \\
         &\Rightarrow A \sqrt{\pi} = \sqrt{\lambda}
         \\
         &\therefore A = \sqrt{\frac{\lambda}{\pi}}
     \end{split}
     \]
     
\end{homeworkProblem}
\pagebreak
\begin{homeworkProblem}
    \textbf{Part (b)}
    \\
    \[
    \begin{split}
         < x > &= \int_{-\infty}^{\infty} x \rho (x) dx
        \\
        &= \int_{-\infty}^{\infty} x A e^{-\lambda (x-a)^2} dx
        \\
        &= A \int_{-\infty}^{\infty} x e^{-\lambda (x-a)^2} dx
        \\
        &= \sqrt{\frac{\lambda}{\pi}} x e^{-\lambda (x-a)^2} dx
    \\
    Let,
    \\
    &u = x-a; \ \ \ \  \ ; x=u+a
    \\
    &\Rightarrow \frac{du}{dx} = 1\\
    \Rightarrow dx = du
    \\
    < x > & = \sqrt{\frac{\lambda}{\pi}} \int_{-\infty}^{+\infty} (u+a) e^{-\lambda u^2} du
    \\
    &= \sqrt{\frac{\lambda}{\pi}} \int_{-\infty}^{+\infty} \left[- \frac{1}{2 \lambda} e^{-\lambda u^2} du + a \int_{-\infty}^{+\infty} e^{- \lambda u^2} du \right]
    \\
    Here,
    \\
    &\int_{-\infty}^{\infty} ue^{-\lambda u^2} du = \int_{-\infty}^{0} ue^{-\lambda u^2} du + \int_{0}^{+\infty} ue^{-\lambda u^2} du
    \\
    Let,
    \\
    v = - \lambda u^2
    \\
    \Rightarrow \frac{dv}{du} = - 2u\lambda
    \\
    \therefore \int - ue^2 \frac{dv}{2u\lambda} &= \int - ue^2 \frac{dv}{2\lambda}
    \\
    &= - \frac{1}{2 \lambda} \int e^2 dv
    \\
    &= - \frac{1}{2 \lambda} e^2
    \\
    &= - \frac{1}{2 \lambda} e^{-\lambda u^2}
    \\
    \end{split}
    \]
    \pagebreak
    \begin{split}
        \int_{-\infty}^{\infty} u e^{\lambda u^2} du = \left[ - \frac{1}{2 \lambda} e^{-\lambda u^2} \right]
    \end{split}
\end{homeworkProblem}
\pagebreak
\begin{homeworkProblem}
    \[
    \begin{split}
        \therefore  \int_{-\infty}^{+\infty} u e^{-\lambda u^2} du &= \left[- \frac{1}{2 \lambda}e^{-\lambda u^2} \right]_{-\infty}^{0} + \left[- \frac{1}{2\lambda} e^{-\lambda u^2} \right]_{0}^{+\infty}
        \\
        &= - \frac{1}{2 \lambda} e^0 - 0 + 0 + \frac{1}{2 \lambda} e^0
        \\
        &= - \frac{1}{2 \lambda} e^0 + \frac{1}{2 \lambda} e^0
        \\
        &= 0
    \end{split}
    \]
    \\
    \\
    \[
    \begin{split}
      \therefore < x > &= \sqrt{\frac{\lambda}{\rho}} [0 + a \cdot \frac{\pi}{\lambda}] 
      \\
      &= \frac{\sqrt{\lambda}}{\pi} \cdot a \cdot \frac{\sqrt{\pi}}{\lambda}
      \\
      &= a
    \end{split}
    \]
    \\
    \\
    \[
    \begin{split}
      < x^2 > &= A \int_{-\infty}^{+\infty} (u+a)^2 e^{-\lambda u^2} du
      \\
      &= A \int_{-\infty}^{+\infty} u^2e^{-\lambda u^2} + 2 a u e^{-\lambda u^2} + a^2 e^{-\lambda u^2} du
      \\
      &= A \left[ \int_{-\infty}^{+\infty} u^2 e^{-\lambda u^2} du + 2a \int_{-\infty}^{+\infty} u e^{-\lambda u^2} du + \int_{-\infty}^{+\infty} e^{-\lambda u^2} du \right]
      \\
      &= A \left[ \int_{-\infty}^{+\infty} u^2 e^{-\lambda u^2} du +
      0 + a^2 \sqrt{\frac{\pi}{\lambda}}\right]
    \end{split}
    \]
    \\\\
    
    \[
    \begin{split}
        We \ know,
        \\
        \frac{d}{dx} \int_{a}^{b} f(x,t) dt &= \int_{a}^{b} \frac{\partial}{\partial x} f(x,t) dt
        \\
        Let,
        \\
        I(\lambda) &= \int_{-\infty}^{\infty} e^{-\lambda u^2} du \ . \ . \ . \ . \ . \ . \ . \ . \ . \ . \ . \ . \ . \ . \ . \ . \ . \ (i)
        \\
        \Rightarrow \frac{dI}{d\lambda} &= \int_{-\infty}^{\infty} \frac{\partial}{\partial \lambda} e^{-\lambda u^2} du
        \\
        &= \int_{-\infty}^{\infty} -u^2 e^{-\lambda u^2} du
    \end{split}
    \]
\end{homeworkProblem}
\pagebreak
\begin{homeworkProblem}
    \[
    \begin{split}
        \textbf{}
        \\
        I(\lambda) = \int_{-\infty}^{+\infty}  e^{-\lambda u^2} du
        \\
        &= \sqrt{\frac{\pi}{\lambda}}
        \\
        \frac{dI}{d\lambda} &= \sqrt{\pi} \frac{d}{d\lambda} (\frac{1}{\lambda})
        \\
        &= \frac{\sqrt{\pi}}{2} \lambda^{-\frac{3}{2}}\\
    \end{split}
    \]
    \\
    \[
    \begin{split}
        So,
        \\
        & - \frac{\sqrt{\pi}}{2} \lambda ^{-\frac{3}{2}} = - \int_{-\infty}^{+\infty} u^2 e^{-\lambda u^2} du
        \\ \\ \\
        \therefore < x^2 > &= A \left[\frac{\pi}{2} \lambda^{-\frac{3}{2}} + a^2 \sqrt{\frac{\pi}{\lambda}}\right]
        \\
        &= \frac{\sqrt{\lambda}}{\sqrt{\pi}} \cdot \frac{\sqrt{\pi}}{2}
        \cdot \lambda^{-\frac{3}{2}} + \sqrt{\frac{\lambda}{\pi}} \cdot a^2 \sqrt{\frac{\pi}{\lambda}}
        \\
        &= \frac{1}{2 \lambda} + a^2
        \\\\
        &\ \ \ \ \ \ \ \ \ \ \ \ \ \ \ \ \ \ \ \ \ \ \ \ \ \ \ \ \ \ \ \ \textbf{[Answer]}
    \end{split}
    \]
    
\end{homeworkProblem}

\pagebreak


%%%%%%%%%%%%%%%%%%%%%%%%%%%%%%%%%%%%%%%%%%%%%%%%%%%%%
%%%%%%%%%%%%%%%%%%%%%%%%%%%%%%%%%%%%%%%%%%%%%%%%%%%%%

                %Solution Number 05%
                
%%%%%%%%%%%%%%%%%%%%%%%%%%%%%%%%%%%%%%%%%%%%%%%%%%%%%
%%%%%%%%%%%%%%%%%%%%%%%%%%%%%%%%%%%%%%%%%%%%%%%%%%%%%
\begin{homeworkProblem}

    \[ \textbf{\underline {Answer to the Question Number Five}}
    \]
    \[
    \textbf{\underline{ Part(a)}}
    \]
\vspace{5mm}
\hspace{20mm}    
\begin{align*}
\large
   &Here,\ given\ a\ wave-function\ of\ a\ particle\ of\ constant\ mass\ m,\\
   & \psi = A exp^{( \frac{-amx^{2}}{\hbar})}\\
   &\therefore \ \ =A e^{( \frac{-amx^{2}}{\hbar})}\ \ \  \ [ notation \ e^{x} \ = \ exp(x)]\\
   &Given\ that\\
   \ \ &\int_{-\infty}^{\infty}\psi . \psi dx = 1\\\\
   &\Rightarrow \int_{-\infty}^{\infty}A e^{( \frac{-amx^{2}}{\hbar})} . A e^{( \frac{-amx^{2}}{\hbar})} dx = 1\\
   &\Rightarrow \int_{-\infty}^{\infty}A^{2} e^{(\frac{-2amx^{2}}{\hbar})} dx = 1\\
   &\Rightarrow A^{2} \int_{-\infty}^{\infty} e^{-(\frac{\sqrt {2am}}{\sqrt{\hbar}}\ .x)^{2}} dx = 1\\
   We\ know, \int_{-\infty}^{\infty} e^{-x^{2}} dx = \sqrt{\pi}\\\\
   Now, \int_{-\infty}^{\infty} e^{-cx^{2}} dx = \frac{1}{c} \int_{-\infty}^{\infty} e^{-x^{2}} dx\\ 
   \Rightarrow \frac{1}{c} \int_{-\infty}^{\infty} e^{-x^{2}} dx  =\frac{1}{c} \sqrt{\pi}\\
   &So,\\
   & A^{2} \int_{-\infty}^{\infty} e^{-(\frac{\sqrt {2am}}{\sqrt{\hbar}}\ .x)^{2}} dx = 1\\\\
   &\Rightarrow A^{2}({\frac{1}{\frac{\sqrt{2am}}{\sqrt{\hbar}}}\ . \sqrt{\pi}}) = 1\\\\
   &\Rightarrow A^{2}{\frac{\sqrt{\pi}}{\sqrt{\frac{2am}{\hbar}}}} =1\\\\
   &\Rightarrow A^{2}\sqrt{\pi}.{\sqrt{\frac{\hbar}{2am}}} =1\\\\
\end{align*}
\newpage\vspace{5mm}
\hspace{20mm}
\large
\begin{align*}
  &\Rightarrow A^{2}  \sqrt{\frac{\pi \hbar}{2am}} =1\\\\
  &\Rightarrow A^{2}=  \sqrt{\frac{2am}{\pi \hbar}}\\\\
  &\Rightarrow A= \sqrt[4]{\frac{2am}{\pi \hbar}}\\\\
  So,\\
  &A =\sqrt[4]{\frac{2am}{\pi \hbar}}\ \ \ [ m, \hbar \ and \ a\ are \ constants ]\\\\
  Answer: A =\sqrt[4]{\frac{2am}{\pi \hbar}}\\\\
\end{align*}
\newpage\vspace{5mm}
    \[
    \textbf{\underline{ Part(b)}}
    \]
\hspace{20mm}
\begin{align*}
\large
  &Here,\\
  &<x> =\int_{-\infty}^{\infty}\psi \hat{x} \psi dx\\
  &<p> =\int_{-\infty}^{\infty}\psi \hat{ p} \psi dx\\\\
  &Now\ computing,\\
  &<x> =\int_{-\infty}^{\infty}\psi x \psi dx\ \ \ [since,\hat{x} = x]\\
  &\Rightarrow <x> =\int_{-\infty}^{\infty} x |\psi|^{2} dx\\
  &= \int_{-\infty}^{\infty}x\ A e^{( \frac{-amx^{2}}{\hbar})} . A e^{( \frac{-amx^{2}}{\hbar})} dx\ \ \ \ [since, \psi =A e^{( \frac{-amx^{2}}{\hbar})}]\\
   &= \int_{-\infty}^{\infty}x\ A^{2} e^{(\frac{-2amx^{2}}{\hbar})} dx \\\\
   let\ substitute,u = \frac{-2amx^{2}}{\hbar}\\
   \Rightarrow du =- \frac{2am}{\hbar}. \ \frac{d}{dx}(x^{2})\\
   \Rightarrow \frac{du}{dx} =-\frac{2(2)amx}{\hbar}\\
   \Rightarrow \frac{du}{dx} = -\frac{4amx}{\hbar}\\
   \Rightarrow dx= -\frac{\hbar}{4amx} du\\
   &so,\\
   &\int_{-\infty}^{\infty}x\ A^{2} e^{(\frac{-2amx^{2}}{\hbar})} dx \\\\
   &=\int_{-\infty}^{\infty}x\ A^{2} e^{u} (-\frac{\hbar}{4amx}) du \\\\
\end{align*}
\newpage\vspace{5mm}
\hspace{20mm}
\large
\begin{align*}   
   &=\int_{-\infty}^{\infty}\ A^{2} e^{u} (-\frac{\hbar}{4am}) du \\\\
   &=-\frac{A^{2}\hbar}{4am}\int_{-\infty}^{\infty}\  e^{u} du\\\\
   &=-\frac{A^{2}\hbar}{4am}\int_{-\infty}^{0}\  e^{u} du -\frac{A^{2}\hbar}{4am}\int_{0}^{\infty}\  e^{u} du\\\\
   &=-\frac{A^{2}\hbar}{4am}\lim_{t\to-\infty} \int_{t}^{0}\  e^{u} du -\frac{A^{2}\hbar}{4am}\lim_{t\to\infty} \int_{0}^{t}\  e^{u} du\\\\
   &=-\frac{A^{2}\hbar}{4am}\lim_{t\to-\infty} [ e^{u} ]_{t}^{0} -\frac{A^{2}\hbar}{4am}\lim_{t\to\infty} [  e^{u}]_{0}^{t}\\\\
   &=-\frac{A^{2}\hbar}{4am}( e^{0} - e^{-\infty} ) -\frac{A^{2}\hbar}{4am}( e^{\infty} - e^{0} )\\\\
   &=-\frac{A^{2}\hbar}{4am}( 1 - e^{-\infty} ) -\frac{A^{2}\hbar}{4am}( e^{\infty} -1 )\\\\
   &=-\frac{A^{2}\hbar}{4am}+ \frac{A^{2}\hbar}{4am} e^{-\infty}  -\frac{A^{2}\hbar}{4am} e^{\infty} +\frac{A^{2}\hbar}{4am} )\\\\
   &= 0\\
   So,\\
   &<x>=0
\end{align*}
\newpage\vspace{5mm}
\hspace{20mm}
\large
\begin{align*}
  &Now\ computing,\\
  &<p> =\int_{-\infty}^{\infty}\psi \ \hat{p}\ \psi dx\\\\
   &<p> =\int_{-\infty}^{\infty}\psi \ -i\hbar\frac{d}{dx}\ (\psi) dx \ \ \ \ [since,\hat{p} =-i\hbar\frac{d}{dx}] \\\\
   here,\\
   &i\hbar\frac{d}{dx}\ (\psi)= i\hbar\frac{d}{dx}(A e^{( \frac{-amx^{2}}{\hbar})})\\\\
   &=i\hbar A e^{( \frac{-amx^{2}}{\hbar})}\frac{d}{dx}(\frac{-amx^{2}}{\hbar})\\\\
   &=i\hbar A e^{( \frac{-amx^{2}}{\hbar})}(-\frac{am}{\hbar}\ . \frac{d}{dx}{x^{2}})\\\\
   &=i\hbar A e^{( \frac{-amx^{2}}{\hbar})}(-\frac{am}{\hbar}\ )2x\\\\
   &=-\frac{i\hbar A e^{( \frac{-amx^{2}}{\hbar})}am2x}{\hbar}\\\\
   &=-2iAamxe^{-\frac{amx^{2}}{\hbar}}\\\\
   So, \ now,\\
   &\int_{-\infty}^{\infty}\psi \ \hat{p}\ \psi dx\\\\
   &=\int_{-\infty}^{\infty}\psi \ -i\hbar\frac{d}{dx}\ (\psi) dx\\\\
   &=\int_{-\infty}^{\infty} A e^{( \frac{-amx^{2}}{\hbar})} \ 2iAamxe^{-\frac{amx^{2}}{\hbar}}\ dx\\\\
   &=2A^{2}iam\int_{-\infty}^{\infty} x e^{( \frac{-amx^{2}}{\hbar})}. \ e^{( \frac{-amx^{2}}{\hbar})} dx
\end{align*}
\newpage\vspace{5mm}
\hspace{20mm}
\large
\begin{align*}
  &=2iamA^{2}\int_{-\infty}^{\infty} x e^{(\frac{-2amx^{2}}{\hbar})} dx\\\\
  &since <x> = A^{2} \int_{-\infty}^{\infty}x\  e^{(\frac{-2amx^{2}}{\hbar})} dx \\\\
  &\Rightarrow<x>=0\ \ \ [we \ got\ by\ computing\ <x>] \\\\
  &therefore,\\
  &<p>=2iamA^{2}\int_{-\infty}^{\infty} x e^{(\frac{-2amx^{2}}{\hbar})} dx\\\\
  &=2iamA^{2} \ . 0 \\\\
  &=0\\\\
  &So, <p> = 0\\\\
  &Answer:<x>=0 ,<p>=0\\\\
\end{align*}
\newpage\vspace{5mm}
 \[
    \textbf{\underline{ Part(c)}}
    \]
\hspace{20mm}
\large
\begin{align*}
  &<x^{2}> =\int_{-\infty}^{\infty}\psi \hat{x^{2}} \psi dx\\
  &<p^{2}> =\int_{-\infty}^{\infty}\psi \hat{ p^{2}} \psi dx\\\\
  &Now\ computing,\\
  &<x^{2}> =\int_{-\infty}^{\infty}\psi x^{2} \psi dx\ \ \ [since,\hat{x^{2}} = x^{2}]\\\\
  &= \int_{-\infty}^{\infty}x^{2}\ A e^{( \frac{-amx^{2}}{\hbar})} . A e^{( \frac{-amx^{2}}{\hbar})} dx\ \ \ \ [since, \psi =A e^{( \frac{-amx^{2}}{\hbar})}]\\\\
   &= \int_{-\infty}^{\infty}x^{2}\ A^{2} e^{(\frac{-2amx^{2}}{\hbar})} dx \\\\
   &= \int_{-\infty}^{\infty}x^{2}\ A^{2} e^{(-{\sqrt{\frac{2am}{\hbar}}}x)}^{2} dx \\\\
   let\ substitute,\\
   u = {\sqrt{\frac{2am}{\hbar}}} x\\\\
   \Rightarrow \frac{du}{dx} = {\sqrt{\frac{2am}{\hbar}}} \ . 1\\\\
   \Rightarrow du = {\sqrt{\frac{2am}{\hbar}}} dx\\\\
   \Rightarrow dx =  {\sqrt{\frac{\hbar}{2am}}} du\\\\
\end{align*}
\newpage\vspace{5mm}
\hspace{20mm}
\large
\begin{align*}
   &here,\\
  & u = {\sqrt{\frac{2am}{\hbar}}} x\\
  &\Rightarrow x {\sqrt{\frac{2am}{\hbar}}}= u\\
  &\Rightarrow x = u \ {\sqrt{\frac{\hbar}{2am}}}\\
  so\ now,\ we \ get,\\
  &<x^{2}>= \int_{-\infty}^{\infty}x^{2}\ A^{2} e^{(-{\sqrt{\frac{2am}{\hbar}}}x)}^{2} dx \\\\
  &= \int_{-\infty}^{\infty}x^{2}\ A^{2} e^{-u^{2}} {\sqrt{\frac{\hbar}{2am}}} \ du  \\\\
  &=  A^{2}{\sqrt{\frac{\hbar}{2am}}} \int_{-\infty}^{\infty}{u^{2}{\frac{(\sqrt{\hbar})^{2}}{(\sqrt{2am})^{2}}}}\ e^{-u^{2}} \ du  \\\\
  &=  A^{2}{\sqrt{\frac{\hbar}{2am}}} \int_{-\infty}^{\infty}{u^{2}{\frac{\hbar}{2am}}}\ e^{-u^{2}} \ du  \\\\ 
  &=  A^{2}{\sqrt{\frac{\hbar}{2am}}}\ {\frac{\hbar}{2am}} \int_{-\infty}^{\infty}u^{2}\ e^{-u^{2}} \ du \ \ \ [equation :01] \\\\
  here,let\ us\ take,
  \int_{-\infty}^{\infty}u^{2}\ e^{-u^{2}} \ du\\\\
  We\ know\\
  Integral\ by\ parts,\\
  \int f dg = fg - \int gdf\\
  let, f = u\\
  \Rightarrow df = du\\
\end{align*}
\newpage\vspace{5mm}
\hspace{20mm}
\large
\begin{align*}
  &And,\\
  &dg = u e^{-u{^2}}\\
  &\Rightarrow g = \int u e^{-u^{2}} du\\
  &let \ substitute, v = -u^{2}\\
  &\Rightarrow dv = -2udu\\
  &\Rightarrow -2u du = dv\\
  &\Rightarrow u du =-\frac{dv}{2}\\
  Here,\\
  &g = \int u e^{-u^{2}} du\ [equation:2]\\
  &=\int- e^{v} \frac{dv}{2} \\
  &=-\frac{1}{2}\int e^{v} dv\\
  &=-\frac{1}{2} e^v \\
  &g =-\frac{1}{2} e^{- u^{2}}+C\\ 
  &when\ putting\ limits\ on\ equation\ 2 \ after\ integration\\
  &=-\frac{1}{2}\lim_{t\to-\infty}[e^{v}]_{t}^{0}-\frac{1}{2}\lim_{t\to\infty}[e^{v}]_{0}^{t}\ \ \ [substituting\ back,-\frac{1}{2} e^v = -\frac{1}{2} e^{- u^{2}}]\\
  &=-\frac{1}{2}\lim_{t\to-\infty}[e^{-u^{2}}]_{t}^{0}-\frac{1}{2}\lim_{t\to\infty}[e^{-u^{2}}]_{0}^{t}\\
  &=-\frac{1}{2}(e^{0} - e^{-\infty^{2}}) --\frac{1}{2}(e^{\infty^{2}} - e^{0})\\
  &=0\\
  &Thus,-\frac{1}{2}e^{- u^{2}} = 0\\
  &So,\ now,\ from\ equation \ 1 \ our\ Integral\ Part \\
  &\int_{-\infty}^{\infty}u^{2}\ e^{-u^{2}} \ du\ \ \ [Integral\ by\ parts,\int f dg = fg - \int gdf]\\\\
  &=-  \frac{ue^{- u^{2}}}{2}- \int_{-\infty}^{\infty} -\frac{e^{- u^{2}}}{2}  du\\
  &=0 + \frac{1}{2} {\sqrt {\pi}} \ \ [we\ know,\int_{-\infty}^{\infty}e^{- x^{2}} =\sqrt {\pi}\ and\ putting \ limits\ on\ equation 02\ ]\\
  &= \frac {\sqrt {\pi}}{2}\\
\end{align*}
\newpage\vspace{5mm}
\hspace{20mm}
\large
\begin{align*}
  &After\ using\ integration\ by\ parts\ and\ now\ putting \ back\ the\ values\ on\ equation 1,\\
  &<x^{2}> =  A^{2}{\sqrt{\frac{\hbar}{2am}}}\ {\frac{\hbar}{2am}} \int_{-\infty}^{\infty}u^{2}\ e^{-u^{2}} \ du\\
  &=A^{2}{\sqrt{\frac{\hbar}{2am}}}\ {\frac{\hbar}{2am}} \frac {\sqrt {\pi}}{2}\\
  &={\sqrt{\frac{2am}{\pi \hbar}}}\ {\sqrt{\frac{\hbar}{2am}}}\ {\frac{\hbar}{2am}}\ \frac {\sqrt {\pi}}{2}\ \  \ [putting \ the\ value\ of \ A^{2}\ from\ Part\ (a)]\\
  &<x^{2}> ={\frac{\hbar}{4am}}\\
  &So,<x^{2}> = {\frac{\hbar}{4am}}\\\\
  &Now\ computing,\\
  &<{p}^{2}> =\int_{-\infty}^{\infty}\psi \ \hat{p^{2}}\ \psi dx\\\\
   &<p> =\int_{-\infty}^{\infty}\psi \ -\hbar^{2}\frac{d^{2}}{dx^{2}}\ (\psi) dx \ \ \ \ [since,\hat{p} =-\hbar^{2}\frac{d^{2}}{dx^{2}}] \\\\
   here,\\
   &\frac{d}{dx}\ (\psi)= \frac{d}{dx}(A e^{( \frac{-amx^{2}}{\hbar})})\\\\
   &= A e^{( \frac{-amx^{2}}{\hbar})}\frac{d}{dx}(\frac{-amx^{2}}{\hbar})\\\\
   &= A e^{( \frac{-amx^{2}}{\hbar})}(-\frac{am}{\hbar}\ . \frac{d}{dx}{x^{2}})\\\\
   &=A e^{( \frac{-amx^{2}}{\hbar})}(-\frac{am}{\hbar}\ )2x\\\\
   &=-\frac{2 Aamx e^{( \frac{-amx^{2}}{\hbar})}}{\hbar}\\\\
\end{align*}
\newpage\vspace{5mm}
\hspace{20mm}
\large
\begin{align*}
   &\frac{d^{2}}{dx^{2}}\ (\psi)= \frac{d^{2}}{dx^{2}}\ (-\frac{2 Aamx e^{( \frac{-amx^{2}}{\hbar})}}{\hbar})\\\\
   &=-\frac{2 Aam}{\hbar}\ . \frac{d}{dx}(\frac{xe^{( \frac{-amx^{2}}{\hbar})}}{{\hbar}})\\\\
   &=-\frac{2 Aam}{\hbar}\ . \frac{d}{dx}(x)\ . e^{( \frac{-amx^{2}}{\hbar})}+x \frac{d}{dx}e^{( \frac{-amx^{2}}{\hbar})}\ \ [Product\ rule]\\\\
   &=-\frac{2 Aam}{\hbar}\ . ( 1 . e^{( \frac{-amx^{2}}{\hbar})}+ (xe^{( \frac{-amx^{2}}{\hbar})}\frac{d}{dx}( \frac{-amx^{2}}{\hbar})))\\\\
   &=-\frac{2 Aam}{\hbar}\ . ( e^{( \frac{-amx^{2}}{\hbar})}+ (xe^{( \frac{-amx^{2}}{\hbar})}( \frac{-am}{\hbar}\ . \frac{d}{dx}{x^{2}})))\\\\
   &=-\frac{2 Aam}{\hbar}\ . ( e^{( \frac{-amx^{2}}{\hbar})}+ (xe^{( \frac{-amx^{2}}{\hbar})} \frac{-am}{\hbar} \ . 2x))\\\\
   &=-\frac{2 Aam}{\hbar}( e^{( \frac{-amx^{2}}{\hbar})} -\frac{2amx^{2} e^{( \frac{-amx^{2}}{\hbar})}}{\hbar}) \\\\
   &=-\frac{2 Aam}{\hbar}(\frac{\hbar e^{( \frac{-amx^{2}}{\hbar})} -{2amx^{2} e^{( \frac{-amx^{2}}{\hbar})}}}{\hbar}) \\\\
   &=\frac{2 Aam}{\hbar}\ . {\frac{(2amx^{2}-\hbar)e^{\frac{-amx^{2}}{\hbar}}}{\hbar}}\\\\
   So, \ now,\\
   &<p^{2}> =\int_{-\infty}^{\infty}\psi \ -\hbar^{2}\frac{d^{2}}{dx^{2}}\ (\psi) dx \\\\
   &=\int_{-\infty}^{\infty}\psi \ (-\hbar^{2}) \frac{2 Aam}{\hbar}\ . {\frac{(2amx^{2}-\hbar)e^{\frac{-amx^{2}}{\hbar}}}{\hbar}}\\\\
   &=\int_{-\infty}^{\infty}\psi \ (-\hbar^{2}) \frac{2 Aame^{\frac{-amx^{2}}{\hbar}}}{\hbar}\ . (\frac{2amx^{2}}{\hbar} -1)\\\\
\end{align*}
\newpage\vspace{5mm}
\hspace{20mm}
\large
\begin{align*}
  &=\int_{-\infty}^{\infty}\psi \ (-\hbar^{2}) -\frac{2 Aame^{\frac{-amx^{2}}{\hbar}}}{\hbar}\ . (1-\frac{2amx^{2}}{\hbar})dx\\\\
  &=\int_{-\infty}^{\infty}\psi \ (\hbar^{2}) \frac{2 Aame^{\frac{-amx^{2}}{\hbar}}}{\hbar}\ . (1-\frac{2amx^{2}}{\hbar}) dx\\\\
  &=\hbar^{2}\int_{-\infty}^{\infty}\psi \  \frac{2 Aame^{\frac{-amx^{2}}{\hbar}}}{\hbar}\ . (1-\frac{2amx^{2}}{\hbar})\\\\
  &=\hbar^{2}\int_{-\infty}^{\infty}\psi \ \psi \ \frac{2 am}{\hbar}\ . (1-\frac{2amx^{2}}{\hbar})\ \ \ [since, \psi =A e^{( \frac{-amx^{2}}{\hbar})}]\\\\
  &=\hbar^{2}\int_{-\infty}^{\infty}\psi^{2}  \ \frac{2 am}{\hbar}\ . (1-\frac{2amx^{2}}{\hbar})\\\\
  &=\hbar^{2}\int_{-\infty}^{\infty}\psi^{2}  \ ( \frac{2 am}{\hbar} -\frac{4a^{2}m^{2}x^{2}}{\hbar^{2}}) dx\\\\
  &= \hbar^{2}\int_{-\infty}^{\infty}\psi^{2}  \  \frac{2 am}{\hbar} dx +\hbar^{2}\int_{-\infty}^{\infty}-\psi^{2}  \ \frac{4a^{2}m^{2}x^{2}}{\hbar^{2}} dx \\\\
  &=\hbar^{2}\frac{2 am}{\hbar}\int_{-\infty}^{\infty}\psi^{2} dx - \hbar^{2} \frac{4a^{2}m^{2}}{\hbar^{2}}\int_{-\infty}^{\infty} \psi^{2} x^{2} dx  \\\\
  &=2am\hbar \int_{-\infty}^{\infty}\psi^{2} dx - {4a^{2}m^{2}}\int_{-\infty}^{\infty} \psi^{2} x^{2} dx  \\\\
  &=2am\hbar \int_{-\infty}^{\infty}\psi^{2} dx - {4a^{2}m^{2}}\int_{-\infty}^{\infty}  x^{2} A^{2} e^{(\frac{-2amx^{2}}{\hbar})} dx \ \  \ [since, \psi =A e^{( \frac{-amx^{2}}{\hbar})}] \\
  & we\ got\ from\ part(a),\int_{-\infty}^{\infty}\psi^{2} dx =1\\
\end{align*}
\newpage\vspace{5mm}
\hspace{20mm}
\large
\begin{align*}
  &and,\ from\ part(c),  \int_{-\infty}^{\infty}x^{2}\ A^{2} e^{(\frac{-2amx^{2}}{\hbar})} dx= {\frac{\hbar}{4am}}\\\\
  &so,\\
  &=2am\hbar \int_{-\infty}^{\infty}\psi^{2} dx - {4a^{2}m^{2}}\int_{-\infty}^{\infty}  x^{2} A^{2} e^{(\frac{-2amx^{2}}{\hbar})} dx\\\\
  &=2am\hbar .1 -4a^{2}m^{2} ({\frac{\hbar}{4am}})\\\\
  &= 2am\hbar -am\hbar\\\\
  &=am\hbar\\\\
  &So,<p^{2}>=am\hbar\\\\
  &Therefore, <x^{2}> = {\frac{\hbar}{4am}}\ and, \ <p^{2}>=am\hbar\\\\
  &Answer:<x^{2}> = {\frac{\hbar}{4am}},\\
  &<p^{2}>=am\hbar\\\\
\end{align*}
\newpage\vspace{5mm}
 \[
    \textbf{\underline{ Part(d)}}
    \]
\hspace{20mm}
\large
\begin{align*}
  &Here,\\
  &{\sigma_{x}}^{2}= <x^{2}> - <x>^{2}\\\\
  &\sigma_{x}=\sqrt{<x^{2}> - <x>^{2}}\\\\
  &=\sqrt{\frac{\hbar}{4am} -0}\ \ \ [since, we\ got \ <x^{2}> \ from\ part\ (c)\ and\ <x>\ from\ part(b)]\\\\
  &=\sqrt{\frac{\hbar}{4am}}\\\\
  &\sigma_{x}=\sqrt{\frac{\hbar}{4am}}\\\\
  And,\\
  &{\sigma_{p}}^{2}= <p^{2}> - <p>^{2}\\\\
  &\sigma_{p}=\sqrt{<p^{2}> - <p>^{2}}\\\\
  &=\sqrt{am\hbar - 0}\ \ \ [since, we\ got \ <p^{2}> \ from\ part\ (c)and\ <p>\ from\ part(b)]\\\\
  &=\sqrt{am\hbar}\\\\
  &\sigma_{p}=\sqrt{am\hbar}\\\\
  &Now,we\ have\ to\ prove,\\
  &\sigma_{x}\sigma_{p}\geqslant \frac{\hbar}{2}\\\\
\end{align*}
\newpage\vspace{5mm}
\hspace{20mm}
\large
\begin{align*}
  &So, \ here,\\
  &\sigma_{x}\sigma_{p}\\\\
  &=\sqrt{\frac{\hbar}{4am}}\ \sqrt{am\hbar}\\\\
  &=\sqrt{\frac{\hbar am\hbar}{4am}}\\\\
  &=\sqrt{\frac{\hbar^{2}}{4}}\\\\
  &=\frac{\hbar}{2}\\\\
  &\sigma_{x}\sigma_{p}=\frac{\hbar}{2}\\\\
  &Thus,\\
  &\sigma_{x}\sigma_{p}\geqslant \frac{\hbar}{2} \ \ \ \ \ \ [Proved]\\\\ 
  &Therefore,\ as,\ \ \sigma_{x}\sigma_{p}=\frac{\hbar}{2},so\ it\ states\ that,\\
  &the\ minimum\ uncertainty\ is\ only\ achieved\ by\ Gaussian\ wave\ function.\\ 
  &Our\ Gaussian\ wave-function\ holds\ Heisenberg’s\ Uncertainty\ principle\\
  &and\ so\ our\ particle’s\ wave-function\ is\ quantum\ mechanical.\\\\\\\\
  &||Thank\ you||
\end{align*}
\end{homeworkProblem}
\pagebreak
















%                       END
%%%%%%%%%%%%%%%%%%%%%%%%%%%%%%%%%%%%%%%%%%%%%%%%%%%%%
%%%%%%%%%%%%%%%%%%%%%%%%%%%%%%%%%%%%%%%%%%%%%%%%%%%%%

%%%%%%%%%%%%%%%%%%%%%%%%%%%%%%%%%%%%%%%%%%%%%%%%%%%%%
%%%%%%%%%%%%%%%%%%%%%%%%%%%%%%%%%%%%%%%%%%%%%%%%%%%%%

\end{document}
